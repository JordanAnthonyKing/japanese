% Created 2021-03-15 Mon 20:16
% Intended LaTeX compiler: pdflatex
\documentclass[11pt]{article}
\usepackage[utf8]{inputenc}
\usepackage[T1]{fontenc}
\usepackage{graphicx}
\usepackage{grffile}
\usepackage{longtable}
\usepackage{wrapfig}
\usepackage{rotating}
\usepackage[normalem]{ulem}
\usepackage{amsmath}
\usepackage{textcomp}
\usepackage{amssymb}
\usepackage{capt-of}
\usepackage{hyperref}
\author{Jordan Cross}
\date{\today}
\title{Organic Japanese lesson notes}
\hypersetup{
 pdfauthor={Jordan Cross},
 pdftitle={Organic Japanese lesson notes},
 pdfkeywords={},
 pdfsubject={},
 pdfcreator={Emacs 28.0.50 (Org mode 9.4)}, 
 pdflang={English}}
\begin{document}

\maketitle
\tableofcontents

My follow-along notes from the Organic Japanese course on Youtube \url{https://www.youtube.com/playlist?list=PLg9uYxuZf8x\_A-vcqqyOFZu06WlhnypWj}

\section{Lesson 1: The core Japanese sentence}
\label{sec:org8ff6ac8}
Every Japanese sentence is fundamentally the same, they have the same core.

Every sentence is formed of two elements, (visualised as a train) they are the \emph{carriage} and \emph{engine}.

In all languages there are only two kinds of sentence:
\begin{itemize}
\item A is B: Sakura is Japanese
\item A does B: Sakura Walks
\end{itemize}

Sakura ga nihonjin da - Sakira is Japanese
Sakura ga aruku - Sakura walks

Going back to the train metaphor:
 \emph{Carriage}      \emph{Engine}
[Sakura *が*] [Nihonjin *だ*]

There is a form of the Japanese sentence core sentence:
\emph{Carriage}  \emph{Engine}
[Pen *が*] [aka *い*] = Pen is red
This is similar, but not identical to an adjective. This will be discussed more later.


To recap, all of these sentences begin with the subject, they are connected with が, and the three \emph{engines} of the Japanese core sentence are:
\begin{itemize}
\item う - verb          - A does B
\item だ - noun          - A is B
\item い - \emph{``adjective''} - A is B
\end{itemize}
\end{document}
