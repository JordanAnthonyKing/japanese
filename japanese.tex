% Created 2021-05-13 Thu 20:20
% Intended LaTeX compiler: pdflatex
\documentclass[11pt]{article}
\usepackage[utf8]{inputenc}
\usepackage[T1]{fontenc}
\usepackage{graphicx}
\usepackage{grffile}
\usepackage{longtable}
\usepackage{wrapfig}
\usepackage{rotating}
\usepackage[normalem]{ulem}
\usepackage{amsmath}
\usepackage{textcomp}
\usepackage{amssymb}
\usepackage{capt-of}
\usepackage{hyperref}
\author{Jordan King}
\date{\today}
\title{}
\hypersetup{
 pdfauthor={Jordan King},
 pdftitle={},
 pdfkeywords={},
 pdfsubject={},
 pdfcreator={Emacs 28.0.50 (Org mode 9.5)}, 
 pdflang={English}}
\begin{document}

\tableofcontents

My follow-along notes from the Organic Japanese course on Youtube: \url{https://www.youtube.com/playlist?list=PLg9uYxuZf8x\_A-vcqqyOFZu06WlhnypWj}
\section{Lesson 1: The core Japanese sentence}
\label{sec:orgcbd8cab}
Every Japanese sentence is fundamentally the same, they have the same core.

Every sentence is formed of two elements, (visualised as a train) they are the \emph{carriage} and \emph{engine}.

In all languages there are only two kinds of sentence:
\begin{itemize}
\item A \textbf{is} B: Sakura is Japanese
\item A \textbf{does} B: Sakura Walks
\end{itemize}

\begin{quote}
Sakura ga nihonjin da - Sakura is Japanese
\end{quote}
\begin{quote}
Sakura ga aruku - Sakura walks
\end{quote}

Going back to the train metaphor:
\begin{center}
\begin{tabular}{ll}
Carriage & Engine\\
Sakura \textbf{が} & Nihonjin \textbf{だ}\\
\end{tabular}
\end{center}

\begin{center}
\begin{tabular}{ll}
Carriage & Engine\\
Pen \textbf{が} & Aka \textbf{い}\\
\end{tabular}
\end{center}

To recap, all of these sentences begin with the subject, they are connected with が, and the three \emph{engines} of the Japanese core sentence are:
\begin{itemize}
\item う - verb          - A does B
\item だ - noun          - A is B
\item い - \emph{``adjective''} - A is B
\end{itemize}

\section{Lesson 2: Invisible が and the を-particle}
\label{sec:orgff47374}
\subsection{The invisible が carriage}
\label{sec:org7f6ff5e}
Part of the reason so many people struggle with Japanese is that, although we can always see the \emph{engine} of the sentence, we cannot always see the が carriage. Remember, the core of the Japanese sentence is formed of the が carriage and an engine.

The closest English equivalent to this invisible が carriage is 'it'. Here is an example English sentence:
\begin{quote}
The ball rolled down the hill. When the ball got to the bottom the ball hit a sharp stone and the ball was punctured and all the air came out of the ball.
\end{quote}

This is not a sentence we would ever say in English, once we have established that we are talking about 'the ball' we would instead refer to the ball by 'it':
\begin{quote}
The ball rolled down the hill. When it got to the bottom it hit a sharp stone and it was punctured and all the air came out of it.
\end{quote}

If we were to completely omit 'it' the sentence would still be easy to understand, we don't \textbf{need} to use this it marker each time, but English grammar \textbf{demands} it. Japanese does not, hence the 'invisible' が carriage.

'It' by itself doesn't really mean anything, we know what it means from context. If a child comes downstairs in the middle of the night and says \emph{'Got really hungry'}, \emph{'Came for something to eat'} we understand that the child means \emph{'I got really hungry'} not \emph{'The dog got really hungry'}. In English this isn't a proper sentence, but in Japanese it is.

All of the little pronouns I/it/we/he/they can all be replaced by the invisible が carriage; the \emph{∅ pronoun}. It is important to remember that the carriage \textbf{is still there}.

One might say: 「ドリーだ」 meaning \emph{'I am Dolly'}. The full sentence being: 「∅がドリーだ」.

By default 'I' is the default value of this \emph{∅ pronoun}. However, if someone were introducing their daughter and said 「ドリーだ」 we would understand from context that ∅ meant this/she.

If I say 「土曜日だ」 - \emph{'It is Saturday'} it is clear that 「∅が」 is \emph{today}.

Each of these sentences are complete grammatical sentences with a subject marked by が and an engine, but in each of these cases the が carriage is just invisible. It \textbf{is} still there. This may seem to be arbitrary, or over-complicated but it saves a lot of grief later on to model sentences this way. Without this information as sentences become more complex they're going to seem increasingly vague and hard to understand.

\subsection{The を-particle}
\label{sec:orgab80e76}
This carriage is formed of a noun and the particle を. The を particle marks the object of the sentence. The thing that some verb (the engine) is being done to (が marks the thing doing the verb). It is not part of the core sentence which is always formed of the が carriage and an engine.

\begin{center}
\begin{tabular}{lll}
Carriage & Carriage & Engine\\
わたし*が* & ケエキ*を* & たべ*る*\\
I & cake & eat\\
\end{tabular}
\end{center}

The core sentence here is 'I eat'. The extra carriage, the を carriage is telling us more about the engine. \emph{What} are we eating? We are eating cake.

Once again, we would often see this said as 「ケーキをたべる」. This is just another case of the invisible が carriage. We \textbf{cannot} have a sentence without a が - we \textbf{cannot} have a sentence without a doer.

When we are saying 「ケーキをたべる」, what we are really saying is 「∅がケーキをたべる」. And the default value for ∅ is 「わたし」 - 'I'.

\section{Lesson 3: は-particle and に-particle}
\label{sec:org3aafc2b}
\subsection{は particle}
\label{sec:org696281a}
The は-particle can never be a part of the core Japanese sentence. It is neither the carriage we are saying something about, nor the engine i.e. what we are saying about it. It isn't a carriage \emph{outside} of the core sentence either like the を-particle is. The は-particle is not part of the logical structure of the sentence.

は is a non-logical particle. In our train metaphor the は-particle is a \emph{flag}. It simply marks something as the topic of the sentence, but doesn't say anything about it.

An exact translation of the は particle would be 'As for \emph{x}'. 「わたしは」 therefore means 'As for me', \textbf{not} 'I am' (わたしが).

A commonly mistranslated sentence is:
\begin{quote}
わたしは日本人だ - I am Japanese
\end{quote}

If we look back at our train however we can see that something is missing:
\begin{center}
\begin{tabular}{ll}
Flag & Engine\\
わたし*は* & 日本人だ\\
\end{tabular}
\end{center}

There is no が carriage. We don't know who the subject actually is. One may ask \emph{'well why don't we just treat the は particle as if it is a carriage'}. In this example it is obvious that the topic marked by は is the same as the subject marked by が, but there are many more cases where this is not true, leading to much confusion down the road. Let's look at a similar sentence. You are at a restaurant, the waitress is asking what you would like:
\begin{quote}
わたしはうなぎだ - \sout{I am an eel}
\end{quote}

Treating は as 'I am' doesn't work. As we now know the default value of the ∅ pronoun is 'I', but in this context it's clear that we're talking instead about \emph{what} we want to eat. 「わたしはうなぎだ」 therefore means 'As for me, eel'.

\subsection{The に particle}
\label{sec:org7d4526c}
The に-particle marks the target (indirect object) of an engine. Along with the が and を we have a sort of \emph{trio} of logical \emph{A does B} sentences:
\begin{itemize}
\item が tells us who does the doing
\item を tells us what it is done to
\item に tells us what the ultimate target of that doing
\end{itemize}

\begin{quote}
わたしがぼーるをなげる - I threw the ball
\end{quote}
The \textbf{core} sentence is 'I threw', and the extra carriage (を) tells us what we threw i.e. the ball.

We can add another carriage to tell us more about the engine:
\begin{quote}
わたしがぼーるをさくら*に*なげる - I threw the ball at/to Sakura
\end{quote}
Sakura is the destination, the target. It is important to note here that the logical particles tell us what happened. The order of the words doesn't really matter the way it does in English.
\begin{quote}
わたし*に*さくらがぼーるをなげる - Sakura threw the ball at/to me.
\end{quote}
\begin{quote}
ぼーるがわたし*に*さくらをなげる - The ball throws Sakura at me
\end{quote}
Obviously this final example doesn't make any sense (although we might want to say something nonsensical like this in a fantasy novel or something) but we can say whatever we like in Japanese so long as we use the right logical particles.

Now let's introduce は:
\begin{center}
\begin{tabular}{llllllll}
flag & carriage &  & carriage &  & carriage &  & engine\\
わたし*は* & ∅*が* &  & さくら*に* &  & ぼーる*を* &  & なげ*る*\\
\end{tabular}
\end{center}
As we know, even if the が carriage is invisible (or silent) this means 'As for me, (I) threw the ball at Sakura'. Now let's give the は \emph{flag} to the ball:
\begin{center}
\begin{tabular}{lllll}
flag & carriage & carriage & carriage & engine\\
ぼーる*は* & わたし*が* & さくら*に* & ∅*を* & なげ*る*\\
\end{tabular}
\end{center}
As for the ball, I threw it (the ball) at Sakura.

This time the を carriage has become invisible, because what we're throwing is now marked by the は particle, ∅ here has taken the value of 'it'. Even without は we might already know what 'it' was that was thrown from context. The important thing to understand here is that as we change the logical particles from one noun to another we change the meaning of the sentence, but when we change the non-logical particle は from one noun to another it makes no difference to the logic of the sentence. It may make some difference to the emphasis, but it makes no difference to who is doing what or what they're doing it to.

\section{Lesson 4: Japanese past, present and future tenses}
\label{sec:orgdd0902d}
Up until now we've only been using one tense and that is the one presented by the plain dictionary form of verbs. To use natural sounding Japanese we need 3 tenses. In Japanese these are not the same past, present and future tenses we're familiar with from English.

The tense we have been using thus far is \textbf{not} the present tense. It is the \emph{non-past} tense. This non-past tense is actually very similar to the English non-past tense. What is the \emph{English} non-past tense? It is again the plain dictionary form of a verb. Eat, run, walk etc. It is unnatural in English to say 'I eat cake', to mean 'I am eating cake'. It is natural however to use the non-past tense to say 'Sometimes I eat cake' or, in the explicitly future tense 'I will eat cake'. Japanese is just the same as English in this way. It is rare we use this form for talking about things actually happening right now, except in cases like literary descriptions.

Most of the time the Japanese non-past tense refers to future events. In fact, just as ∅ defaults to 'I', the non-past tense defaults to the future.
\begin{quote}
さくらが歩く - Sakura will walk
\end{quote}
\begin{quote}
犬がたべる - Dog will eat
\end{quote}
The way we have been using this tense up until now i.e. \emph{'Sakura walks'}, is possible, but isn't the most natural way.

If we want to say something more natural like \emph{'Sakura is walking'} we must use the verb \emph{'to be'}\footnote{In English the verb 'to be' is irregular and has multiple forms be/is/are/am: To \emph{be} walking, Sakura \emph{is} walking, not Sakura \emph{be} walking.}, or in Japanese 「いる」:
\begin{quote}
さくらが歩いている - Sakura is walking
\end{quote}
\begin{quote}
犬がたべている - Dog is eating
\end{quote}
There is something here however that we haven't yet seen. In our train metaphor this is a \emph{secondary engine}, here 「たべて」 which could be an engine in of itself, is helping (modifying) the main 「いる」 engine. Our core sentence is still the same, we have a が carriage and an engine, 「いる」 i.e. 「さくらがいる」 - Sakura is (existing). The secondary engine modifies 「いる」 telling us more about what state she is currently existing in, she is in the eating state. As we go further into Japanese we will see this secondary engine structure again and again.
\begin{center}
\begin{tabular}{lll}
carriage & secondary-engine & engine\\
犬*が* & たべて & い*る*\\
\end{tabular}
\end{center}

Also, just as in English we don't say 'The dog is eat', we use a special form of the verb \emph{eat} => \emph{eating}. In Japanese this is the て form. This is covered in the next lesson.

For the past tense of verbs instead of adding て to verbs we add た.
\begin{quote}
犬がたべた - The dog ate
\end{quote}
The way in which we do this is exactly the same as the way in which we attach て and will be covered in the next lesson.

If we want to make it clear that we are talking about a future event we can add a time expression. By prefixing a sentence with あした (tomorrow), we can make it clear that what we will be doing, we will be doing tomorrow.
\begin{quote}
あした∅がケーキをたべる - Tomorrow I will eat cake
\end{quote}
Note how we simply preface the sentence with 'tomorrow', just like we would in English. This is the case with all \emph{relative-time nouns}, 'yesterday', 'tomorrow', 'the day after tomorrow', 'next week', next month', 'next year'.

For \emph{non-relative}, i.e. \emph{absolute} time expressions we must use the に-particle:
\begin{quote}
火曜日*に*∅がケーキをたべる - On Tuesday I will eat cake
\end{quote}
We must attach に in all the same places we would attach on/in/at in English. 'On Tuesday', 'in March', 'at 12 o'clock'. Fortunately in Japanese we only need to use the one particle.

\section{Lesson 5: Japanese verb groups and て-form}
\label{sec:orgec98248}
Japanese verbs fall into three groups: \emph{Ichidan}, \emph{Godan}, and \emph{irregular}

The first group are \emph{ichidan} (lit: one level) verbs. Morphing these verbs is easy, we simply remove the る and add our new ending. Ichidan verbs can only end in either いる or える.

The second group is by far the largest, the \emph{godan} (lit: five level) verbs. This group contains verbs that end in all of the possible verb endings: う つ る ー ぬ ぶ む ー く ぐ ー す. Each of these ending groups has its own way of being morphed, though although they're 'five level' verbs, two of the groups use the same method so we only need to learn 4 methods. Confusingly this means that godan verbs can end in いる or える, most of these will still be \emph{ichidan} verbs, and fortunately even if a verb is morphed incorrectly, you will probably still be understood.
\begin{itemize}
\item う つ る -> って
\item ぬ ぶ む -> んで
\item く/ぐ -> いて/いで (Note: this is the combined group)
\item す -> して
\end{itemize}

There are only two irregular verbs, くる and する. いく, is partly irregular, but not completely.
\begin{itemize}
\item くる -> きて
\item する -> して
\item いく -> いって (\sout{いいて})
\end{itemize}
These are the only exceptions
\section{Lesson 6: Japanese ``adjectives''}
\label{sec:org25c24a5}
\subsection{い-adjectives, verb adjectives, and な-adjectives}
\label{sec:org0841271}
Japanese adjectives are not the same as English adjectives. As we have learned Japanese sentences come in three kinds, depending on the type of engine they have. As a reminder they are:
\begin{itemize}
\item う - verb - A \textbf{does} B
\item だ - noun - A \textbf{is} B
\item い - ``adjective'' - A \textbf{is} B
\end{itemize}

The truth is that all three of these types of engines can be used like adjectives.

Let's start with the first one, the one we refer to as an adjective in English, the い-engine:
\begin{quote}
ぺんがあかい - Pen is red
\end{quote}
An important note, 「あかい」 does not mean 'red', it means 'is red'. 「あか」 means red.

If we swap the order of 「ぺんが」 and 「あかい」 then we can take this い-engine, and now use it not as the primary engine, but as a secondary engine. This would not be a complete sentence however without a new engine, for example, a new (primary) い engine.
\begin{quote}
あかいぺんがちいさい - Red pen is small
\end{quote}
This is simple enough, let's take a look at verbs.

Any う (verb) engine, in any tense can be used like an adjective:
\begin{quote}
しょうじょがうたった - Girl sang
\end{quote}
\begin{quote}
うたったしょうじょが - The girl who sang (Note: this sentence is not yet complete, it lacks a primary engine).
\end{quote}
\begin{quote}
うたったしょうじょがねている - The girl who sang is sleeping
\end{quote}

Next, the noun engine:
\begin{quote}
いぬがやんちゃだ - The dog is naughty
\end{quote}
We can turn 「やんちゃ」 into an adjective too, but there is one important thing to note. Just as we have to add だ to a noun, here we must add な to the noun. な is the connective form of だ. Don't be fooled by 'な-adjectives', they're simply nouns!
\begin{quote}
やんちゃないぬが - The dog who is naughty (Note: this sentence is not yet complete, it lacks a primary engine).
\end{quote}
\begin{quote}
やんちゃないぬがねている - The dog who is naughty is sleeping
\end{quote}

An important note is that we cannot do this with \emph{all} nouns, only nouns which are frequently used in an adjectival way. This group of nouns is what the are referred to as 'な-adjectives'. We can use all nouns as adjectives, but for the rest we need to use a different technique and for that we will have to learn about the の particle.

\subsection{The の-particle}
\label{sec:org0a56dae}
The の particle, or the \emph{possessive particle} functions just like the English \emph{'s}.
\begin{quote}
さくら*の*はな - Sakura's nose
\end{quote}
\begin{quote}
わたし*の*はな - Me's (my) nose
\end{quote}
Luckily in Japanese we don't have to worry about his/her/my/their, we just use の.

Because this is the \emph{possessive particle} we can use this in another slightly different way. 「あか」 has an \emph{adjectival} form in 「あかい」, but not all colours have this form. The Japanese for pink, 「ピンクいる」 (lit: pink-colour) doesn't have an adjectival form in 「ピンクいろい」, nor can we use it as a secondary engine with な. So what are we to do? Well we can use the の-particle:
\begin{quote}
ピンクいる*の*どれすが - The pink dress (literally: The dress belonging to the class of pink things)
\end{quote}
\begin{quote}
うさぎ*の*OSCAR - Oscar the rabbit (literally: Oscar belonging to the class of rabbit)
\end{quote}

Just as before, there's no need to worry about misusing の and な, no-one listening is going to misunderstand what you're saying and it's a very typical beginner mistake to make.

Using these techniques we can make all kinds of sentences that can become very complex, especially with verbal adjectives in which we can use whole sentences in an adjectival manner.

\section{Lesson 7: Negatives and adjective ``conjugations''}
\label{sec:orgb26167a}
\subsection{Negative nouns}
\label{sec:org3c724d3}
The fundamental basis of negatives is the adjective 「ない」. This adjective means 'non-exist'. The word for exist (for any inanimate thing) is ある. If we want to say that something exists:
\begin{quote}
ぺんがある
\end{quote}
But if we want to say that something doesn't exist we say:
\begin{quote}
ぺんがない
\end{quote}
Now, why do we use a verb for being, and an adjective for non-being? This is something that happens all throughout Japanese, when we do something we use a verb, but when we don't do something we attach ない and are therefore using an adjective as the engine of the sentence. This has a very logical reason, when we \textbf{do} something, an action is taking place, and so we use a verb, but when we \textbf{don't do} it we are describing a state of non-action, so that's an adjective.

Above we have said 'There is no pen', how do we say 'This is not a pen'?
\begin{quote}
これはぺんがある - As for this, it's a pen
\end{quote}
\begin{quote}
これはぺんではない - As for this, as for being a pen, it's not. (Note: で is the て-form of だ)
\end{quote}

\subsection{Negative verbs and the Japanese stem system}
\label{sec:orga10268c}
Now let's look at negative verbs. To make a verb negative we must attach 「ない」 to the あ-stem of the verb. How do we do this?

Note: these stems apply to Godan verbs. For Ichidan verbs we simply drop the る and add 「ない」. Remember, all ichidan verbs end in る but not all る ending verbs are ichidan verbs.
\begin{quote}
たべる ー> たべない
\end{quote}

Here is the kana-grid, presented on its side. Every verb ends in one of the う-row kana. (う-row kana that aren't used as verb endings have been removed).

\begin{center}
\begin{tabular}{lllll}
あ & い & \textbf{う} & え & お\\
か & き & \textbf{く} & け & こ\\
さ & し & \textbf{す} & せ & そ\\
た & ち & \textbf{つ} & て & と\\
な & に & \textbf{ぬ} & ね & の\\
ば & び & \textbf{ぶ} & べ & ぼ\\
ま & み & \textbf{む} & め & も\\
ら & り & \textbf{る} & れ & ろ\\
\end{tabular}
\end{center}

As we can see there are four other ways in which the verb could end. These are the verb stems. For now we're only looking at the あ-stem as this is the one we need for the negative.

To from the あ-stem we simply shift the final kana from the う-row to the あ-row. There is one only exception and this is the only exception in the entire stem system. This exception is that う itself does not become あ but わ. This is because, take for example a verb like 「かう」, 「かあ」 would not be as easy to say as 「かわ」. Every other う-row kana is simply changed to its あ-row equivalent.
\begin{center}
\begin{tabular}{lllll}
\textbf{\textbf{わ}} & い & \textbf{う} & え & お\\
\textbf{か} & き & \textbf{く} & け & こ\\
\textbf{さ} & し & \textbf{す} & せ & そ\\
\textbf{た} & ち & \textbf{つ} & て & と\\
\textbf{な} & に & \textbf{ぬ} & ね & の\\
\textbf{ば} & び & \textbf{ぶ} & べ & ぼ\\
\textbf{ま} & み & \textbf{む} & め & も\\
\textbf{ら} & り & \textbf{る} & れ & ろ\\
\end{tabular}
\end{center}

So to form the negative form of a verb convert it to the あ-stem and add ない.
\begin{quote}
かう ー> かわない
\end{quote}
\begin{quote}
はなす ー> はなさない
\end{quote}

\subsection{Negative adjectives and adjective ``conjugations''}
\label{sec:org0d09f08}
The adjective stem is simple, just drop the い and add く. This is how we make the て form, 「あかい」 ー> 「あかくて」, and it's also the way we make the negative, 「あかい」 ー> 「あかくない」.

If we want to put an adjective into the past tense we drop the い and add かった.
\begin{quote}
こわい ー> こわかった - Was scary
\end{quote}

Because 「ない」 is also an adjective, the past tense of it is just 「なかった」.
\begin{center}
\begin{tabular}{lll}
Non-past & Past & \\
\hline
さくらがはしる & さくらがはしった & Positive\\
さくらがはし*ら*ない & さくらがはし*ら*なかった & Negative\\
\end{tabular}
\end{center}

Now as we know 「さくらがはしる」 is not very natural Japanese, instead we would say 「さくらがはしっている」. For this, all we need to do is put the 「いる」 into the past tense:
\begin{quote}
さくらがはしっている -> さくらがはしっていた - Sakura was running
\end{quote}

\subsection{The only exceptions}
\label{sec:orgc95aa94}
There are only two real exceptions to what has been covered in this lesson. They are the helper verb 「ます」 which makes words formal by adding it to the い-stem of a verb. When we put 「ます」 into the negative it does not become 「まさない」 as we would expect, but becomes 「ません」, because it is formal it is a bit old-fashioned and uses the old Japanese negative 「せん」 instead of 「ない」.

The only other exception is いい (is good), which has an older form, which is still widely used in よい. When we morph いい it becomes よい again:
\begin{quote}
いい ー> よくない - Not-good
\end{quote}
\begin{quote}
いい ー> よかった - Was-good
\end{quote}
\begin{itemize}
\item Note: 「よかった」 is a common phrase: 「∅がよかった」 - \emph{It was good (That went well, it turned out great etc.)}
\end{itemize}

\section{Lesson 8: Location, purpose and transformation (に and へ particles).}
\label{sec:orgc8f4c79}
We already know that in a logical sentence the に-particle marks the ultimate target of an action. If we are going somewhere, or sending something somewhere, or putting something somewhere, we use に.
\begin{quote}
∅がみせ*に*いく - (I) will go to the shop
\end{quote}
We can also mark a more subtle kind of target:
\begin{quote}
∅がみせ*に*たまごをかい*に*いく - (I) will go to the shop to buy eggs
\end{quote}
\begin{itemize}
\item Note: かい is the い stem of かう, to buy.
\end{itemize}

If we recall, the logical particles (が, に, を) can only mark nouns. The い stem of a verb is the equivalent noun form of it. Just as in English 'I like swimming', \emph{swimming} is a noun, 'I go to the shop for the purpose of buying eggs', this \emph{buying} is also a noun.

に gives us the target of an action in the literal sense, and also the target in a volitional sense, i.e. the aim of our action.

As well as identifying a place we will go to, に can also specify a place we are currently at:
\begin{quote}
∅がみせ*に*いる - (I am) at the shop
\end{quote}
This に is still marking a target, just not a future target. In order for something to be somewhere it must've gotten there, and so に specifies the target of some past action. We can also use this for inanimate objects:
\begin{quote}
ほんは∅がテーベルのうえ*に*ある - As for the book, (it) exists (is) on-at the table.
\end{quote}
\begin{itemize}
\item Note: うえ is a noun, meaning the on/top of something.
\end{itemize}

Finally, に can also mark a transformation. If \emph{a} becomes \emph{b}, then に also marks \emph{b}, the thing a is becoming.

\begin{quote}
さくらは∅がかえる*に*なった - Sakura became a frog
\end{quote}
Of course this example is a bit of a joke, but there are of course various every day things that become other things. This form of expression is also used much more often in Japanese than in English.
\begin{quote}
ことし∅が十八さいになる - This year (I) become 18 years old
\end{quote}
\begin{quote}
あとで∅がくもりになる - Later (it (the weather) will) become cloudy
\end{quote}

For adjectives things work slightly differently:
\begin{quote}
さくらがうつくしい - Sakura is beautiful
\end{quote}
If we want to say 'Sakura became beautiful' we can't use に because うつくしい isn't a noun, (referring back to our metaphor), it's not a carriage, it's an engine. All we need to do is turn the adjective into its stem by removing い and adding く refering back to lesson 7:

\begin{verbatim}
The adjective stem is simple, just drop the い and add く. This is how we make the て form, 「あかい」 ー> 「あかくて」, and it's also the way we make the negative, 「あかい」 ー> 「あかくない」.
\end{verbatim}
\begin{quote}
さくらがうつくしくなった - Sakura became beautiful
\end{quote}

\subsection{The へ car}
\label{sec:orgf432ded}
Note: when used as a particle へ is pronounced え.

This is a very simple particle, it duplicates a single use of に. When we say \emph{a is going to b} we can freely substitute に with へ. This is \textbf{all} is can do, it cannot even mark the case where something \emph{is}, only where it is \emph{going}.
\section{Lesson 8b: Japanese particles explained}
\label{sec:orgbc37e3d}
A logical particle tells us how the sentence logically holds together. It tells us who does what to whom with what, when and where.

は is a non-logical particle, it simply identifies the topic, but doesn't say anything about it. Other particles like と are \emph{alogical}, they aren't simply markers. In the case of と the particle 'ands' two nouns together. It is therefore doing something in the sentence, in our train metaphor it is joining a noun-carriage to another carriage, inheriting its logical particle, but has no function of its own.
\begin{quote}
さくら*と*メイリー*が*あるいていた - Sakura \textbf{and} Mary were walking
\end{quote}

Logical particles \textbf{always} attach to a noun. If we see a logical particle attached to anything else then we know that that word is functionally a noun.

The noun and the particle attached to it are an inseparable pair. We must view the two together, they are a question and an answer which form a fundamental unit of the Japanese sentence.

There cannot be a sentence without が, even if sometimes you can't see it. が can work in A is B sentences, descriptive sentences. The other particles can \textbf{only} work in A does B sentences, that is sentences with a verb engine.
\begin{itemize}
\item が - Who (or what) did it?
\item を - Whom was it done to?
\item に - Where did they go?/Where are they?
\item へ - What direction?
\item で - Where was it done? With what was it done?
\end{itemize}
\begin{quote}
∅がこうえん*に*いる - I am in the park
\end{quote}
\begin{quote}
∅がこうえん*で*あそんでいる - I am playing in the park
\end{quote}
Remember, 「あそんで」 is the て form of 「あそぶ」 and is a secondary engine, modifying 「いる」. \emph{I am} -> \emph{I am playing}.
\begin{quote}
∅がこうえん*に*いく - I go to the park
\end{quote}
\begin{quote}
∅がバス*で*こうえんいいく - I go to the park by bus
\end{quote}
If we say 'I went by bus' or 'I ate with chopsticks' we use で for the thing we did it with, the means by which we performed the action.
\section{Lesson 9: 'Ego-centrism' and expressing desire}
\label{sec:org8289833}
\subsection{Ego-centrism}
\label{sec:org2a189e6}
English is an \emph{ego-centric} language. Japanese is a more \emph{animist} language. What this means is that English always wants a person, preferably \emph{I}, but if not I then someone else, or perhaps it will settle for an animal, but always wanting an animate being to be acting. Japanese is not this way:
\begin{quote}
わたしはコーヒがすきだ
\end{quote}
A textbook would translate this as 'I like coffee'. 'I like coffee' very well might be the English equivalent for this simple phrase, but it is not what this sentence means, and it is not what this structure of sentence means.

The が is marking the coffee. The coffee is the actor in this sentence, not I. It's not 'I like coffee', I am not \emph{liking} it. But 'As for me, coffee is likeable/pleasing'.

The English 'I like coffee' is an \emph{A does B} sentence. The Japanese is an \emph{A is B} sentence.

すき is a \textbf{noun}. An adjectival noun, but still a noun. It is not a verb like in English.

If this sentence were 1-1 with the common English meaning given, then every single part of it would be miss-described by the particles.
\begin{itemize}
\item は does not mark an actor
\item が does not mark an object
\item だ does not mark a verb
\end{itemize}

\begin{quote}
わたしはほんがわかる
\end{quote}

\subsection{Expressing desire}
\label{sec:org5aef43c}
\begin{quote}
わたしは(optional)こねこがほしい - As for me, a kitten is wanted
\end{quote}
「ほしい」 is often translated in English as 'want', but again, it is not a verb, it is an adjective. Again, \emph{I} is not the actor of the sentence, it is the cat, and it is the cat that is wanted, not 'I want a cat'.

In Japanese the way that wanting to \textbf{do} something is expressed is different to the way that wanting to \textbf{have} something is expressed.

The way this is done is with the い-stem again. To express that we want to do something, we must add the \emph{helper adjective} 「たい」 to the い-stem of the verb. 「たい」 doesn't mean 'want' in the English sense, it can't, again, because \textbf{want} is a verb, and 「たい」 is an \textbf{adjective}.
\begin{quote}
(わたしは)クレープガたべたい
\end{quote}
The common English translation for this is 'I want to eat crepes', but as we see the pattern is just the same as in the other cases, the desire-ability of the crepes is not a verb, it is an adjective.

There is no truly good translation of this into English. We shouldn't be thinking in terms of 'awkward English' or 'natural English' when it comes to constructing and understanding these sentences. We should be thinking in terms of Japanese. The 'awkward' translations of the Japanese are only there to give a \textbf{grasp} of the structure of Japanese.

Now, what if we took this sentence, 「わたしはクレープガたべたい and removed the optional parts so that we just had 「たべたい? In this case, the meaning of the sentence would be what the common English translation is. ∅ defaults to I, and so the translation is 'I eat-wanting am' -> 'I am wanting to eat' -> 'I want to eat'. Because there is no eat-inducing subject here, the want to eat is attributed directly to I.

So what is 「たい」? Is it an adjective describing the \emph{condition of something} making you want to do something, or is it an adjective describing \emph{my desire}? Well, it can be either. This is very common throughout Japanese. 「こわい」 can mean scared or scary:
\begin{quote}
おばけがこわい - Ghosts are scary
\end{quote}
\begin{quote}
∅がこわい - I am scared
\end{quote}
This isn't confusing because が tells us what to do.

\subsection{A final note to help keep things clear}
\label{sec:orge196ee5}
We cannot use these adjectives of desire (or any emotion) about anyone other than ourselves. If we say 「たべたい」 and there is no context to give the subject, then we must be talking about ourselves, and never the person we are speaking to or anyone else. Japanese simply doesn't allow us to use 「たい」 or 「こわい」 or 「ほしい」 or anything else about anyone other than ourselves.

If we wanted to say that someone else wants something then, because Japanese is such a logical language it doesn't allow us to say something that we cannot know for sure. One thing that we cannot know for sure is someone's inner feelings. We might think that Sakura wants to eat cake, but we can't know for sure. So if I want to talk about her desire to eat cake, we can't just use 「たい」. We need to add to 「たい」 (or 「こわい」, or 「ほしい」 or anything else) the helper verb がる.

To do this we take the い off of the adjective and add the helper verb 「がる」.
\begin{itemize}
\item たがる
\item こわかる
\item ほしがる
\end{itemize}

「がる」 means 'to show signs of', 'to look as if it's the case'.
\begin{quote}
さくらがケーキをほしがる - Sakura is showing signs of wanting cake
\end{quote}
Even if Sakura has actually told me she wants cake, we must still use 「がる」. All I know is what she's said, I still don't know her feelings absolutely.

Why do we use a verb for other people and an adjective for ourselves? I can't describe someone-else's feelings because I don't know about them, I can only describe their actions, and their actions are a verb.
\section{Lesson 10: Japanese ``conjugation'' and potential form}
\label{sec:org0d21a2e}
What textbooks typically refer to as ``conjugations'' are really just \emph{helper-verbs} attached to the four verb stems.

We have already looked at the helper-adjectives 「ない」 and 「たい」, as well as the helper-verb 「がる」. Now for the potential helper-verb which attaches to the え-stem of a verb.
\begin{center}
\begin{tabular}{lllll}
あ & い & \textbf{う} & \textbf{え} & お\\
か & き & \textbf{く} & \textbf{け} & こ\\
さ & し & \textbf{す} & \textbf{せ} & そ\\
た & ち & \textbf{つ} & \textbf{て} & と\\
な & に & \textbf{ぬ} & \textbf{ね} & の\\
ば & び & \textbf{ぶ} & \textbf{べ} & ぼ\\
ま & み & \textbf{む} & \textbf{め} & も\\
ら & り & \textbf{る} & \textbf{れ} & ろ\\
\end{tabular}
\end{center}

The potential helper verb has two forms, for godan verbs 「る」, and for ichidan verbs 「られる」.

There are only two exceptions, 「くる」 and 「する」.
\begin{quote}
くる -> こられる
\end{quote}
\begin{quote}
する ー> できる
\end{quote}

There is only one difficult spot with the potential form, and it is similar to the issue from the previous lesson.
\begin{quote}
わたしは(optional)ほんがよめる - As for me, the book is readable
\end{quote}
A common translation of this would be 'I can read the book', however again the が is on the book, not on I. If we wanted to say 'I can read the book', the book would need to be marked by を as it is the target of our reading, and 'I' would have to be marked by が as I is the actor.
\begin{quote}
わたしがほんをよめる - I can read the book
\end{quote}
This is perfectly fine, but it's not what is usually done. Remember, Japanese is not \emph{ego-centric}.

As we're using a helper-verb, the past, non-past, negative-past and negative-non-past conjugation rules are the same as regular verbs, for 「あるける」 (can walk):
\begin{itemize}
\item あるける - non-past
\item あるけた - past
\item あるけない - negative-non-past
\item あるけなっかた - negative-past
\end{itemize}
\section{Lesson 11: Compound sentences, くれる, あげる, and more て form uses}
\label{sec:orge0d5284}
\begin{quote}
ある 日 アリスは 川の そばに いた。
\end{quote}
そば is a noun meaning 'beside', so 川のそば means beside the river.

ある means 'a certain', so ある日 means 'on a certain day'. Notice that here we're using a verb as an adjective, as we have described in previous lessons. 本がある (a book exists) -> ある本が (an existing/a certain book). Note also that this is the same as how we use might use Today, Yesterday, Tomorrow etc. but not how we might use 'On Saturday'.
\begin{quote}
On a certain day, alice was beside a river.
\end{quote}

\begin{quote}
おねえちゃんは つまらない 本を よんで いて あそんで くれなかった。
\end{quote}
よむ (read) -> よんでいる (reading) -> よんでい*て* - We have put the いる into the て form, why have we done this?
\begin{quote}
おねえちゃんは つまらない 本を よんで いる - Big sister is reading an uninteresting book.
\end{quote}
This by itself is a complete clause (sentence), by turning the engine of the sentence (いる) into the て form we're saying that something else is going to follow this clause, i.e. 'and'.
\begin{quote}
おねえちゃんは つまらない 本を よんで いて - Big sister is reading an uninteresting book and\ldots{}
\end{quote}
\begin{quote}
あそんで くれなかった
\end{quote}
あそぶ is to play. This has also been put into the て form. Here we have another use of the て form, let's examine. くれる means to 'give downwards' i.e. as Japanese is so polite we place ourselves below others, so someone else is giving us something. あげる to contrast is to 'give upwards' i.e. to give to someone else. What is being given? In this case the thing being given is what is attached to it via the て form, i.e. 'playing'. Specifically, she is not giving the act of playing to Alice. In Japanese we frequently use 'give' for actions, for doing something for our benefit as well as for literally 'giving' nouns. If someone does something for our benefit, we turn that action to the て form, and attach it to くれる. If we do something for someone else's benefit, we turn that action to the て form, and attach it to あげる.
\begin{quote}
Big sister is reading an uninteresting book and didn't play with Alice (didn't play for Alice's benefit).
\end{quote}

Note again our two clauses:
\begin{quote}
おねえちゃんは つまらない 本を よんで いて
\end{quote}
\begin{quote}
あそんで くれなかった
\end{quote}
For the first clause we do not know what in what tense the action is taking place. In English we would place the tense marker on both clauses, in Japanese we only do this at the end. よんで いて could mean 'is reading' and it could mean 'was reading'. Because くれなかった is in the past tense, then the entire sentence is in the past tense.
\section{Lesson 12: と quotation particle and compound verbs and compound nouns}
\label{sec:orgc553e54}
\begin{quote}
「おもしろい ことが ない」 と アリスは 言った
\end{quote}
\begin{itemize}
\item もの - thing concrete
\item こと - thing abstract
\end{itemize}

\begin{quote}
「おもしろい ことが ない」- No interesting (abstract) thing exists (Nothing interesting is going on here)
\end{quote}
\begin{quote}
アリスは 言った - Alice said
\end{quote}
The intesting thing to note is the particle between these two, the と particle. There are two と particles, one means 'and', and one marks a quotation. When we quote someone as saying something or as thinking something we use this と particle. We also use these square brackets which are the equivalent of English quotation marks, but in speech we cannot see these, so we also use と (and clearly use と in writing regardless also).
\begin{quote}
「おもしろい ことが ない」 と アリスは 言った - Nothing interesting is happening said Alice
\end{quote}

\begin{quote}
そのとき、白い ウサギが とおり すぎた。
\end{quote}
\begin{itemize}
\item そのとき - That time, in this sentence it is used to mean 'just at that moment' (just as Alice said that)
\item とおる - pass through
\item すぎる - exceed, go beyond
\end{itemize}

とおりすぎる is doing something very interesting. It is attaching the い stem of one verb, to another verb to give it extra meaning. We will see this a lot throughout Japanese. Connecting とおる and すぎる, 'pass through' and 'go beyond' means 'passing by'.
\begin{quote}
そのとき、白い ウサギが とおり すぎた。 - At that moment a white rabbit passed by
\end{quote}

\begin{quote}
ふつうの ウサギでは なくて、 チョッキを きて いる ウサギ だった。
\end{quote}
\begin{quote}
ふつうの ウサギでは なくて - Not an ordinary rabbit
\end{quote}

\begin{quote}
チョッキを きて いる ウサギ だった。- It was a rabbit that was wearing a vest (it was a wearing a vest rabbit)
\end{quote}


\begin{quote}
ウサギは かいちゅうどけいを 見て 「おそい!おおい!」と言って、はしり だした。
\end{quote}
\begin{itemize}
\item かいちゅうどけい is the combination of かいちゅう (inside pocket) and どけい (watch) meaning pocketwatch. An interesting note here is that とけい becomes どけい (I may write up a full explanation of why anothe time).
\end{itemize}

\begin{quote}
「おそい!おおい!」と
\end{quote}
What と does structurally, is it takes whatever it marks which could be a whole paragraph, or just two short words like this, or anything with all sorts of grammar going on, it takes whatever it marks as a quotation and turns it into a single noun. Going forward we will find that this is used not only to mark things people say and people think, but to mark all sorts of things. This と structure can therefore make a quotation act as a modifier to whatever follows, in this cast it is modifying 言う (to say), or to think or feel, but could be many things.

Note: As there must be a ∅ somewhere the rabbit is either saying 'it's late' or 'I'm late'.
Note also, we don't need to use ウサギは again as it is already marked in the first half of the sentence.
\begin{quote}
はしり だした - Run + Take out = Broke into a run (started to run). In this sense だした means modifies the verb to mean the action 'erupted'.
\end{quote}
\begin{quote}
ウサギは かいちゅうどけいを 見て 「おそい!おおい!」と言って、はしり だした。 - The rabbit looked at his watch and said 'I'm late! I'm late' and broke out into a run.
\end{quote}
Note: all the uses of て-form to mark 'and'.
\begin{quote}
「ちょっと まって ください」 と アリスは よんだ。
\end{quote}
\begin{quote}
「ちょっと まって ください」 と - Please wait a little
\end{quote}
\begin{quote}
「ちょっと まって ください」 と アリスは よんだ。
\end{quote}
よんだ is the て-form of よぶ (to shout/call).
Note: Both よむ and よぶ conjugate to よんだ in the て form, fortunately it's not likely that we'll get these two verbs mixed up.
\begin{quote}
「ちょっと まって ください」 と アリスは よんだ。 - Please wait a minute called Alice
\end{quote}

\begin{quote}
でも ウサギは ピョンピョンと はしり つづけた。
\end{quote}
\begin{itemize}
\item でも - but
\item はしる + つづける = Continued running (running continued).
\item ピョンピョン - The sound of a small thing jumping along
\end{itemize}

Once again we're using the quotation particle と to describe the way in which it run, it ran in the way it sounds, it ran like a small thing jumping along (note there are no quotation marks around this).
\section{Lesson 13: Passive ``conjugation'' - Not passive and not a conjugation}
\label{sec:org447f62b}
The real name for the 'passive conjugation' is the \emph{Receptive helper verb}.

The receptive helper verb is れる for godan verbs and られる for ichidan verbs, and attaches to the あ-stem of another verb.
\begin{center}
\begin{tabular}{lllll}
\textbf{あ} & い & \textbf{う} & え & お\\
\textbf{か} & き & \textbf{く} & け & こ\\
\textbf{さ} & し & \textbf{す} & せ & そ\\
\textbf{た} & ち & \textbf{つ} & て & と\\
\textbf{な} & に & \textbf{ぬ} & ね & の\\
\textbf{ば} & び & \textbf{ぶ} & べ & ぼ\\
\textbf{ま} & み & \textbf{む} & め & も\\
\textbf{ら} & り & \textbf{る} & れ & ろ\\
\end{tabular}
\end{center}
Remember! う becomes わ, not あ.

The receptive helper verb means \emph{receive} or \emph{get}, we're receiving/getting the action that the helper verb is attached to
\begin{quote}
さくらがしか*ら*れた - Sakura scolded-got - Sakura got scolded/Sakura received a scolding
\end{quote}

Note, the receptive helper verb and the modified verb have different actors. The sentence is not Sakura scolds, someone else (we don't know who) is scolding Sakura, but Sakura is the one in the act of receiving the scolding. This is not the same with all helper verbs.

The receiver is not always a person:
\begin{quote}
水がの*ま*れた - Water got drunk
\end{quote}
Even if we add a doer of the drinking, the water is still the actor of the sentence.
\begin{quote}
水がいぬにの*ま*れた - Water got drunk by (a) dog
\end{quote}
いぬに is modifying のむ.

Why is the dog being marked by に? Let's look at a larger sentence:
\begin{quote}
さくらは だれかに かばんが ぬす*ま* れた - As for Sakura, someone-by bag stolen-got - As for Sakura, (her) bag got stolen by someone
\end{quote}

Who is the actor? It's not Sakura, she's marked by は. It's not the 'someone' as they're marked by に. The bag is the actor of the sentence, the bag \emph{did} 'got'.

What is に doing here? に marks the ultimate target of an action. So what is the target of getting stolen? To whome is the stolen item going? It is the 'someone' who stole it.

Note: Cure Dolly uses a \emph{push-pull} analogy here, which I think is unnecessary.

\subsection{The nuisance receptive}
\label{sec:orgdda694b}
\begin{quote}
さくらが だれかに かばんを ぬす*ま* れた
\end{quote}
Here the core of the sentence is now 'Sakura got'. What did she get? She got the unfortunate (nuisance) action of だれかに かばんを ぬすむ, someone stealing (her) bag. \emph{Sakura got her bag stolen by someone} \textbf{not} \emph{Sakura's bag got stolen by someone}.
\section{Lesson 14: Adverbs and も-particle}
\label{sec:org8123ec4}
\begin{quote}
アリスは とび 上がって、 ウサギの 後を 追った。
\end{quote}
\begin{itemize}
\item とび上がって = とぶ (jump/fly) + 上がる (rise up) = Jump up
\item 後を追った = 後 (after) + 追う (follow) = Follow after/follow behind
\end{itemize}

Note how we are following ウサギの後 - \emph{the rabbit's after/behind}. We are following the back of the rabbit.
\begin{quote}
アリスは とび 上がって、 ウサギの 後を 追った。 - Alice jumped up and followed after the rabbit
\end{quote}

---

\begin{quote}
しゃべる ウサギを 見た ことが ない。
\end{quote}
Here しゃべる is being used as an adjective just as any verb can be. しゃべるウサギ - Talkative/talking rabbit.

見た is the passed tense of 見る to see. It is modifying こと, an abstract thing, meaning 'The fact of having seen'. 見たことがない means 'The fact of having seen doesn't exist'.

The talking rabbit is the object of the engine of the sentence, 'The fact of having seen'. So: 'The fact of having seen a talking rabbit doesn't exist' -> (Alice) had never seen a talking rabbit. This is another example of the un-egocentric nature of Japanese; Alice is not the actor of this sentence, it is the 'thing' that does not exist.

---

\begin{quote}
ウサギは 早く 走って、 急に ウサギの 穴に とび 込んだ。
\end{quote}
\begin{quote}
ウサギは 早く 走って、
\end{quote}
\begin{itemize}
\item 早い - fast/early
\item 走る - run
\end{itemize}
早い is an adjective. If we want to say the rabbit is fast we simply say ウサギが早い. But if we want to say that the rabbit's \textbf{movement} is fast we must use an adverb. In Japanese we can turn any adjective into an adverb by simply removing the い and replacing it with く. 早い -> 早く.

Note again how our verb 走る has been converted to the て form signifying an 'and'.

\begin{quote}
急に ウサギの 穴に とび 込んだ。
\end{quote}
\begin{itemize}
\item ウサギの 穴に - Here we aren't saying 'The rabbit's hole' but 'a hole of the variety rabbit'.
\item とび 込む = とぶ (jump) + こむ (Squeeze (?) into) = Jump into
\end{itemize}

急に is another way of forming an adverb. 急 is a noun meaning 'suddenly'. We can form an adjective from a noun by adding に.

\begin{quote}
ウサギは 早く 走って、 急に ウサギの 穴に とび 込んだ。- The rabbit ran quickly and jumped into a rabbit hole.
\end{quote}

\subsection{The も flag}
\label{sec:org1f596b5}
\begin{quote}
アリス*も* ウサギの 穴に とび こんだ。 - Alice also jumped into the rabbit hole.
\end{quote}

も is another non-logical topic-marking particle. も marks the topic of the sentence in the same way that は does. The difference is that while は can mark the topic of the sentence and \textbf{can} also change the topic of the sentence, も decalares the topic of the sentence but can \textbf{only} change the topic of the sentence. We cannot use も unless we are changing the topic of the sentence. Up until this point the topic of our conversation has been the rabbit, now we are switching to talk about Alice.

When we change topic with も we're saying that the comment about the previous topic (the rabbit and that it jumped) is the same as our new topic (Alice). When we change the topic with は we are doing the opposite, we are drawing a distinction between the two.

---

\begin{quote}
穴の 中は たて穴 だった。 アリスは すぐ下に 落さた。- The inside of the hole was a vertical hole. Alice fell straight (directly) down the hole.
\end{quote}

\begin{quote}
でも、 おどろいたことに ゆっくり ゆっくり 落さた。 - But, the surprising thing was that she slowly slowly fell/ But, surprisingly she fell slowly.
\end{quote}
\begin{itemize}
\item おどろいたこと doesn't mean 'A surprised thing', it means 'the surprising thing' (surprisingly). The に attached is again to turn it into an adverb. So: 'She fell surprisingly'. Of course, it isn't surprising that she fell, but it is suprising that she fell ゆっくり ゆっくり (slowly slowly).
\item ゆっくり is slightly unusual in that it is fundamentall a noun, but we can use it as an adjective without adding に to it. We will see ゆっくり very often.
\end{itemize}

\section{Lesson 15: Transitivity}
\label{sec:orgaa884c2}
\emph{Transitive} and \emph{intransitive} aren't as big of a misnomer as some of the things we've seen so far, but a better pair of terms would be \emph{Self-move} and \emph{Other-move}.

In japanese, a move-word 動詞 (どうし) is a word that denotes an action of a movement. So a self-move verb is a verb that moves itself. If I 'stand-up' that's a self-move action. But throwing a ball is an 'other-move' action, one is not throwing themselves, they are throwing a ball. It's as simple as that.

Japanese has a lot of pairs of words, these could be called forms, or just closely related words, that give the self-move and other-move variations of the verb. For example:
\begin{itemize}
\item 出る (でる) - leave, exit, come out - Self-move
\item 出す (だす) - take out, bring out - Other-move
\end{itemize}

Most of the time we can tell which is a self-move word and which is an other-move word by following a few simple rules.

The first thing to know is that there is a root word for self-move and a root word for other-mode:
\begin{itemize}
\item ある - Be - Self-move
\item する - Do - Other-move
\end{itemize}

Knowing this there are three laws of move-word pairs.
\begin{enumerate}
\item す and せる (え-stem) ending verbs are other-move
\item あ-stem + る (aru) ending verbs are self-move
\item え-stem + る (eru) flip self/other-move either way
\end{enumerate}

Honorary members of the す family:
\begin{itemize}
\item む -> める is always other-move
\item ぶ -> べる is always other-move
\item つ -> てる is always other-move
\end{itemize}

The only wildcards left are:
\begin{itemize}
\item く/ぐ -> ける/げる
\item う ー> える
\item Some る-ending verbs not covered by the first two laws
\end{itemize}

Is there anything we can do to simplify this: える version have the opposite of the standard word.

\section{Lesson 16: て-みる, 'try doing', や-particle, から-particle, exclusive-'and'}
\label{sec:org7d83cd6}
\begin{quote}
落ちる 間に ひもが たっぷり あって まわりを ゆっくり 見まわせた。
\end{quote}
\begin{itemize}
\item 落ちる - fall
\item 間に - during
\item ひま - spare time
\item たっぷり - in plenty
\item まわり - surroundings
\item 見まわす - look around
\end{itemize}

\begin{center}
\begin{tabular}{lllll}
verb working as adjective for 間 &  &  & adverbial noun not needing に & \\
Secondary-\textbf{う} engine & noun & core & adverb & \textbf{う}-engine\\
落ち*る* & 間*に* & ひま*が* & たっぷり & あって\\
fall & during & free time & plentifully & existed-and\\
\end{tabular}
\end{center}

\begin{quote}
まわりを ゆっくり 見まわせた。
\end{quote}
\begin{itemize}
\item 回る - Go around (self-move)
\item 回す - Send/Make go around (other-move)
\item まわり - Noun form of 回る meaning 'surroundings', not the act of going around
\end{itemize}

\begin{center}
\begin{tabular}{lllll}
 & い-stem info &  & い-stem info & まわる + potential helper + past tense\\
core & object & adverbial noun & secondary-\textbf{う} engine & \textbf{う}-engine\\
∅*が* & まわり*を* & ゆっくり & 見 & まわせた\\
Alice & surroundings & leisurely & lookage- & around-send-could\\
\end{tabular}
\end{center}

\begin{quote}
落ちる 間に ひもが たっぷり あって まわりを ゆっくり 見まわせた。- While (she was) falling a lot of spare time existed, and (she) could look leisurely around her surroundings.
\end{quote}

---

\begin{quote}
まずは、下を 見てみた けど、暗すぎて 何も 見えなかった。
\end{quote}
\begin{itemize}
\item まず - first
\item 下 - down
\item 暗い - dark
\item すぎる - exceed
\item 何も - even-anything
\end{itemize}

見てみた: When we add みる to the て-form of another verb we're saying to 'try' doing something. We're literally saying 'do it and see'. 食べてみろ = Try eating (it) and see. やってみろ = do and see (give it a try). 見てみた = Take a look and see.

\begin{quote}
まずは、下を 見てみた
\end{quote}
\begin{center}
\begin{tabular}{lllll}
Relative time expression & core & object & secondary-\textbf{う} engine & \textbf{う}-engine\\
まず*は* & ∅*が* & 下を & 見て & 見た\\
first of all & Alice & down & look- & tried\\
\end{tabular}
\end{center}

\begin{quote}
暗すぎ*て* - (it was) dark exceeded-and - It was too dark and therefore
\end{quote}

\begin{quote}
何も 見えなかった - even anything as for (it) did-not-do-see-able - It was too dark to see anything
\end{quote}

---

\begin{quote}
その後、穴の まわりを 見て、目に 止まるのは ぎっしり ならんだ とだな や 本だな だった。
\end{quote}

\begin{quote}
その後、穴の まわりを 見て - After that she looked at the surroundings of the hole and\ldots{}
\end{quote}

\begin{quote}
目に 止まるのは - Eye-at as for stopped-thing - The thing that stopped her eye - The thing that caught her eye
\end{quote}

\begin{quote}
その後、穴の まわりを 見て、目に 止まるのは - After that she looked at the surroundings of the hole and the thing that caught her eye was\ldots{}
\end{quote}

\begin{quote}
ぎっしり ならんだ とだな や 本だな だった。- tightly lined-up cupboards-and bookshelves-was
\end{quote}

\textbf{The や-particle}

When putting to clauses together we use the て-form as an equivalent to 'and'. When putting two things together we can use the と (exclusive 'and') and や (non-exclusive 'and') particles.

---

\begin{quote}
たなの 一つから びんを 取り下した。- (She) shelve's one-from jar take-lowered - From one of the shelves she took down a jar.
\end{quote}

\section{Lesson 17: Form Japanese: です/ます + volitional}
\label{sec:orgbb24562}
ます is a (helper) verb that attaches to the い stem of another verb. It doesn't change the meaning of the verb in anyway, it just makes it formal. ます is highly irregular:
\begin{itemize}
\item The past tense is normal, it ends just like any す verb - ました
\item The past tense is not ませない, it is ません - This is the only verb that does this in modern Japanese
\item The negative past is ませんでした
\end{itemize}

です is the formal version of だ and works exactly the same aside from one quirk, unlike with だ which we do not attach to adjectives, we do attach です in formal speech. It doesn't mean or do anything, but it's done all the same.

A useful note, we can use ません and ないです interchangeably:
\begin{itemize}
\item さくらが話しません
\item さくらが話しないです
\end{itemize}

\textbf{The volitional form}

The volitional form is one of the few uses of the お-stem. The godan volitional helper is just う, attached to the お stem it simply lengths the お sound. The ichidan form is to as usual remove the る, and add よう.

The volitional form of ます and です are ましょう and でしょう.

\begin{center}
\begin{tabular}{lllll}
あ & い & \textbf{う} & え & \textbf{お}\\
か & き & \textbf{く} & け & \textbf{こ}\\
さ & し & \textbf{す} & せ & \textbf{そ}\\
た & ち & \textbf{つ} & て & \textbf{と}\\
な & に & \textbf{ぬ} & ね & \textbf{の}\\
ば & び & \textbf{ぶ} & べ & \textbf{ぼ}\\
ま & み & \textbf{む} & め & \textbf{も}\\
ら & り & \textbf{る} & れ & \textbf{ろ}\\
\end{tabular}
\end{center}

Volition means will, the volitional form expresses or invokes the will of the speaker. The most usual use of it is setting the will of a group of people in a particular direction. いきましょう = Let's go.

There are many uses of the volitional form in combination with various particles but they will be covered in due time. One of note for now is the volitional copula だろう / でしょう which when added to any ordinary sentence adds the extra meaning of \emph{probably}, i.e. it adds doubt/conjecture.
\begin{itemize}
\item 赤いでしょう - Probably red
\item さくらがくるでしょう - Sakura's probably coming.
\end{itemize}

\section{Lesson 18: Trying to do something; って = は? として、 と言う/という、 と する、 おう と する、 っていう}
\label{sec:org9ee6740}
\subsection{Try - と する}
\label{sec:orgb360580}
\begin{quote}
山にのぼろうとする - Try to climb the mountain
\end{quote}
\begin{itemize}
\item のぼる = Climb
\item Note: に is normal here because we are climbing to the \emph{target} of the summit.
\end{itemize}

Why does this mean 'try'? A precise translation may be derived from のぼろう - have the will to climb, and する to do. I have the will to climb the mountain, and I will do it, (but I may not have the ability). Hence, \emph{try}.

\subsection{How we regard something - と する}
\label{sec:org66cbc0f}

See again how we're using the と quotation particle once again. と is encapsulating, not the words or thoughts of someone, but the meaning of 「山にのぼろう」 and putting that meaning into action (する).

Another example:
\begin{quote}
ホッとする
\end{quote}
\begin{itemize}
\item ホッ is the sound effect for a sigh of relief
\end{itemize}

We aren't saying here that someone breathed a sigh of relief. What we're saying is that they \emph{enacted what was expressed by the sound effect} i.e. They were relieved. In 山にのぼろうとする we're enacting the feeling of setting out to climb the mountain.

\begin{quote}
∅が さくらを \textbf{日本人と} する - We Sakura ``Japanese person'' enact - We assume/take Sakura to be Japanese
\end{quote}
We are thinking/acting according what is expressed by the quote.

Compare this with:
\begin{quote}
∅が さくらを \textbf{日本人に} する - We turned Sakura into a Japanese person
\end{quote}

\begin{quote}
かばんを \textbf{まくらと} する - Use bag as a pillow
\end{quote}
Not literally: turn bag into a pillow.

\subsection{See something in the light of being something - と して}
\label{sec:orgbe45143}
The closest equivalent to this in English is 'as'. i.e. 'My opinion \emph{as} a private person'.
\begin{quote}
会長*として* - As President\ldots{}
\end{quote}

We can also use it as 'for':
\begin{quote}
アメリカジン*として*小さい - She's small for an American. (As an American, she's small).
\end{quote}

\subsection{As a quotation -  と言う/という}
\label{sec:orga49c324}
The most basic thing that can follow と is 言う in which case it can be used as a literal quotation of something that's been said (as we've already seen) but can also be used as a way of saying how something is said or what it's called:
\begin{quote}
ふしぎの国のアリスという本 - The book called 「ふしぎの国のアリス」
\end{quote}
\begin{itemize}
\item Note: という is usually written in Kana when it precedes something
\end{itemize}

という can be reduced down as far as just って. という -> っていう -> って:
\begin{quote}
ふしぎの国のアリスっていう本 - The book called 「ふしぎの国のアリス」
\end{quote}
\begin{quote}
ふしぎの国のアリスって本 - The book called 「ふしぎの国のアリス」
\end{quote}

\subsection{って as は}
\label{sec:org2003c66}
Remember the は particle is the topic marking particle:
\begin{quote}
さくらは ∅が 日本人だ - As for Sakura, (she) is a Japanese person
\end{quote}
\begin{quote}
さくらって ∅が 日本人だ - Speaking of Sakura, (she) is a Japanese person
\end{quote}
This is a very casual use, we can't use という in place of は but we can use って. The point is that this is still very logical.

\section{Lesson 19: Causative + 'causative passive'}
\label{sec:org6c3759a}
The \emph{causative helper verb} indicates that we are causing someone to do the verb to which is is attached.

The \emph{receptive helper verb} indicates receiving the action to which it is attached.

The causative and the receptive both attach to the あ-stem of a verb.
\begin{itemize}
\item Causative: せる/させる
\item Receptive: れる/られる
\end{itemize}

\begin{center}
\begin{tabular}{lllll}
\textbf{わ} & い & \textbf{う} & え & お\\
\textbf{か} & き & \textbf{く} & け & こ\\
\textbf{さ} & し & \textbf{す} & せ & そ\\
\textbf{た} & ち & \textbf{つ} & て & と\\
\textbf{な} & に & \textbf{ぬ} & ね & の\\
\textbf{ば} & び & \textbf{ぶ} & べ & ぼ\\
\textbf{ま} & み & \textbf{む} & め & も\\
\textbf{ら} & り & \textbf{る} & れ & ろ\\
\end{tabular}
\end{center}

It is important to understand that the verb and the helper verb always have different subjects
\begin{quote}
水が 犬に 飲む れた - The water got drunk by the dog (receptive)
\end{quote}
\begin{itemize}
\item Core action: verb れる (get/receive), actor 水 (water)
\begin{itemize}
\item Secondary action: verb 飲む (drink), actor 犬 (dog)
\begin{itemize}
\item Implicit sub-clause: 犬が 飲んだ
\end{itemize}
\end{itemize}
\end{itemize}

\begin{quote}
さくらが だれかに かばんを ぬすま れた - Sakura got her bag stolen by someone (receptive)
\end{quote}
\begin{itemize}
\item Core action: verb れる (get/receive), actor さくら
\begin{itemize}
\item Secondary action: object かばん (bag), actor だれか (someone)
\begin{itemize}
\item Implicit sub-clause: だれかが かばんを ぬすんだ
\end{itemize}
\end{itemize}
\end{itemize}

\begin{quote}
∅が 犬を 食べ させた - I caused the dog to eat (causative)
\end{quote}
\begin{itemize}
\item Core action: verb させる (cause), actor ∅ (I)
\begin{itemize}
\item Secondary action: verb たべる (eat), actor 犬 (dog)
\begin{itemize}
\item Implicit sub-clause: 犬が 食べた
\end{itemize}
\end{itemize}
\end{itemize}

せる/させる can mean to \emph{compel/make/force}, or it can mean to \emph{allow}. But it can also mean neither of those. The best way to translate it is with the rather non-native sounding 'cause'. 'I caused the dog to eat' doesn't mean 'I forced the dog to eat', it just means that I did something that had the result of the dog eating, whether that be intentional, accidental, or forceful.

Sometimes the person or thing we are causing to do something can be marked by を and sometimes it can be marked by に. Remember the particles are always consistent. If we are forcing someone to do something, then we're treating them like an object (を), if we're treating them as a target, then this is more mutual and goes with \emph{allowing} over \emph{compelling} and so に.

That said the use of を and に is not the main indicator of if we mean allowing or compelling. There is no precise English analogy with the causative helper verb, so trying to determine if it means exactly cause or exactly allow is misguided, remember, it can mean neither. It's sort of a sliding scale between the two, more subtle. Beyond that, when the action that is being compelled has itself an を marked object we can see that in the sub(ordinate) sentence the meat is the object of the dog's action, and the dog is the thing that is being caused to do the action:
\begin{quote}
∅が犬に にくを 食べ させた - I caused the dog to eat meat
\end{quote}
\begin{itemize}
\item Core action: verb させる (cause), actor ∅ (I)
\begin{itemize}
\item Secondary action: verb たべる (eat), actor 犬 (dog)
\begin{itemize}
\item Implicit sub-clause: 犬が にくを 食べた
\end{itemize}
\end{itemize}
\end{itemize}

に expresses relation to core clause. を expresses relation to sub clause.

In these types of sentences Japanese does not allow us to use the を particle twice. If we could use を twice then in some sentences we might end up with some doubt as to which を marks the object associated with 食べる and which を marks the object associated with せる/させる.

\subsection{Causative receptive (causative passive)}
\label{sec:orgeaaa5e9}
Causative-receptive (what most call the causative passive) means to get made to do. Remembering that helper verbs are ichidan verbs, to add the receptive helper verb to the causative helper verb we simple remove the る and add られる:
\begin{itemize}
\item せる/させる - る + られる = せられる/させられる
\end{itemize}

We now have three verbs in a sentence:
\begin{quote}
わたしは ∅が ブロコリを 食べ(1) させ\textsuperscript{2} られた\textsuperscript{2} - I got\textsuperscript{3} compelled\textsuperscript{2} to eat\textsuperscript{1} broccoli
\end{quote}
\begin{itemize}
\item Core action: verb られる (get), actor ∅ (I)
\begin{itemize}
\item Secondary action: verb させる (compel), actor unspecified
\begin{itemize}
\item Tertiary action: verb 食べる (eat), actor ∅ (I)
\end{itemize}
\end{itemize}
\end{itemize}

Note: The first and third actor are always the same. The second actor always different.
\section{Lesson 20: Sore/Sono/Sonna/Sou etc. Directional Words}
\label{sec:org9bf48df}
こ そ あ ど words initially simply mark physical locations, but then expands out to more subtle and metaphorical uses. This is common because all languages use physical metaphors to express abstract concepts.

The most basic meaning:
\begin{center}
\begin{tabular}{llll}
ここ & そこ & あそこ & どこ\\
Here (near me) & There (near you/a little way off) & Over there & Where?\\
\end{tabular}
\end{center}
\begin{itemize}
\item Often ここ means the speaker's place and そこ means the listener's place
\item Often あそこ means away from both the speaker and the listener
\end{itemize}

\subsection{れ-group (nouns)}
\label{sec:orgda567a6}
The れ group act as nouns.

\begin{center}
\begin{tabular}{llll}
これ & それ & あそれ & どれ\\
This & That & That (other there) & Which thing?\\
\end{tabular}
\end{center}

These can get confused with the の-group as in English we use the same word for both of these types of words.

\subsection{の-group (adjectivals)}
\label{sec:org2c1fc73}
The の group act as adjectives.

\begin{center}
\begin{tabular}{llll}
これ & それ & あそれ & どれ\\
This-something & That-something & That-something (other there) & \\
described as near me & described as near you & described as over there & how-described?\\
\end{tabular}
\end{center}

れ means a being, it refers to a thing. の is used to make adjectivals and descriptors:
\begin{quote}
これは (∅が) ペンだ - As for here-thing (it) pen-is - The thing here is a pen
\end{quote}
\begin{quote}
この ペンは (∅が) 赤い - Here's pen as for (it) red-is - The pen that is here is red
\end{quote}

\subsection{な-group (real adjectives)}
\label{sec:org16b4199}
The な-group act as real adjectives

\begin{center}
\begin{tabular}{llll}
こな & そな & あそな & どな\\
Like this & Like that & Like that & Like what?\\
\end{tabular}
\end{center}

The な used for adjectival nouns descriptive of a thing's qualities. Distance is often conceptual, no physical. i.e. How far the thing is from what we're talking about or the present-circumstance.
\begin{quote}
こんあ食べ物 = Food like this
\end{quote}
\begin{quote}
そんあことがひどい - A thing like that is cruel
\end{quote}

These are essentially comparing-adjectives. Saying that something is like something either in physical space or in a conceptual way.

\subsection{う/あ-group}
\label{sec:orgf6cb633}
Lengthening the final sound of こ/そ/あ/ど is talking about the way something is/happens.

\begin{quote}
∅がそうです - It (the fact/situation) is that way = That's right
\end{quote}
\begin{quote}
そうせる - Do like that
\end{quote}
\begin{quote}
こうせる - Do like this
\end{quote}
\begin{quote}
どうせる - Do it like how?
\end{quote}

\begin{quote}
どうすればいい - In what way if I act will be good?
\end{quote}
\begin{itemize}
\item すれば is the conditional form of する
\end{itemize}

\begin{quote}
そういうこと - That way say matter (condition/thing) - Thus-described matter - That kind of thing
\end{quote}
\begin{quote}
どういうこと - What way said matter (condition/thing) - What is going on here?
\end{quote}
The いう here is not referring to the fact we've said anything. The いう refers to the description of the thing.
\section{Lesson 21: Te oku/te aru}
\label{sec:org172585b}
\begin{quote}
その びんいは ラベルが 貼って あって 「オレンジ•マーマレード」と 書いて あった
\end{quote}
\begin{quote}
その びんいは ラベルが 貼って あって - A label was pasted to the bottle
\end{quote}
貼って ある is something not yet covered. We've used the て-form of a verb + いる meaning to be in the state of doing that verb. て ある also means to be in the state of that verb, however there is a difference:
\begin{quote}
窓が開いている - The window is open
\end{quote}
\begin{quote}
窓が開けてある - The window is open
\end{quote}
What is the difference? いる simply means is open, however ある carries another implication. Notice the use of the other-move verb in ある and the self-move verb in いる. 開けてある therefore signals that the window is open \textbf{because} someone opened it.

Notice how we're using いる, the verb for animate objects to describe the openness of an inanimate object. Because we have used the self-move verb the inanimate object is an honorary 'willed being' with a state of its own. In the example with ある the state has been caused externally, and so the window maintains its inanimateness.

\begin{quote}
その びんいは ラベルが 貼って あって - The jar existed in the state of having a label stuck to it and\ldots{}
\end{quote}

\begin{quote}
「オレンジ•マーマレード」と 書いて あった - (it) was in the state of having 'Orange marmalade' written on it
\end{quote}

\subsection{ておく}
\label{sec:orgdf77a5e}
\emph{Skipping ahead a little} Alice realises that the marmalade jar is in fact empty. What is she to do with it? She doesn't want to drop it, what will she do?
\begin{quote}
[アリスが びんを] うまく とだなの 一つく 通りすがりに 置いて* おいた* - She skilfully in passing put the jar into one of the cupboards
\end{quote}
\begin{itemize}
\item * Same word used twice, helper usually written in kana alone
\item うまく - skilfully
\item とだな - cupboard
\item 通りすがり - in passing
\item 置く - place/put
\end{itemize}

The second おく here (ておく) is in a sense the second half of てある:
\begin{quote}
窓が開けてある - Exist in the state of having been made open
\end{quote}
\begin{quote}
窓を開けてあく - Open the window so that it remains in the state of openness. Establish the window as being in the state of openness
\end{quote}
In many cases this is used to mean 'doing something in advance' but it is not the only meaning, as we see here. What is literally means is putting the action in place.
\section{Lesson 22: Te-wa, te-mo - Topic/comment magic}
\label{sec:org85ca4b5}
\begin{quote}
でも、びんは 空っぽ だった - But the jar was empty
\end{quote}
---
\begin{quote}
アリスは 空っぽの びん*でも*、下へ 落としては 悪いと 思った
\end{quote}
The で here is the て form of だ and it is attached to も, the inclusive \emph{and} particle and the reverse subject particle of は:
\begin{quote}
∅が 空っぽの びんでも - It empty jar-is as-much-as
\end{quote}
\begin{itemize}
\item も = as-much-as i.e. 'even though'
\end{itemize}
The て-form + topic-marker combination forms a complete logical clause that is subordinate to (requiring comment from) the following clause
\begin{quote}
∅が 空っぽの びんでも - Even though it was an empty jar\ldots{}
\end{quote}

This is a new usage of も, there are many similar usages of this particle:
\begin{itemize}
\item いちま円もかかた掛かる - He took as much as 10,000 yen
\item ケエキを食べてもいい - Is it alright if I eat this cake? lit. If I go as far as to eat the cake is that alright?
\end{itemize}

\begin{quote}
下へ 落としては 悪いと 思った
\end{quote}
落としては now we have て from + は topic marker. While も is the additive, including particle, は is the subtractive, excluding particle. So, while も means 'as much as', は means 'as little as'. We tend to use でも in positive contexts, and では in negative contexts. Often this ては gets contracted into just ちゃ.

We can use ては as the connector between two clauses, and it implies that the second clause is unwanted:
\begin{quote}
雨が 降っ*ては* : ∅が こうえんい 行けない - Rain falls and (negative comment expected) : we park-to can-go-not
\end{quote}
This is much the same as how も is followed by a comment, here we follow は with a comment explaining why the rain falling is a bad thing.
\begin{quote}
∅が いもうとと けんかし*ては* : ははに しかられた - I sister-with quarrelled-and (negative result) : mother-by scolded-got
\end{quote}

ても on the other hand doesn't indicate a positive result or a negative result, it indicates a contrasting result:
\begin{quote}
雨が 降っ*ても* : ∅が こうえんい 行く - Even though it's raining we can still go to the park
\end{quote}

This is where でも itself, rightly translated as 'but', comes from; でも wraps up whatever came before it literally meaning 'as much as [that] is [so]\ldots{}'. でも is the all purpose ても contrast marker.

\begin{quote}
アリスは 空っぽの びんでも、下へ 落としては 悪い*と* 思った
\end{quote}
と bundles the entire thing into a quotation, of Alice's thoughts (思て). The full sentence therefore is: 'Empty jar-is-even-though, downward drop (negative expectation) is bad, Alice thought.' - 'Alice thought that even though the jar is empty, dropping it would be bad'.

\section{Extra: も particle combinations}
\label{sec:orgd844265}
も is the additive particle, and can also be used to mean 'as much as' in several expressions. In English 'as much as' can also be said as 'even'. The same is true in Japanese.
\begin{itemize}
\item 誰もない - Not even someone
\item 何もない - Not even something
\item 少しもない - Not even a little
\end{itemize}

What about でも? The で here is not the particle で, it is the て-form of だ. As we know だ (the copula) couples together two nouns. What is it coupling? It is couple the sentence that came before it and the ∅ pronoun (it). So 誰でも doesn't mean 'everyone', it means anyone.
\begin{itemize}
\item 誰でも - Anyone - 誰でもできろ - Even if it's anyone, they can do it - Anyone can do it
\item 何でも - Anything - 何でもいい - Even if it's anything, it's good - Anything is good
\item どうでも - However (Whatever way) - どうでもいい - Whatever way is fine
\end{itemize}

\begin{quote}
かもしれません - Perhaps
\end{quote}
\begin{itemize}
\item か - Makes a question of whatever came before
\item も - Even/As much as
\item 知れ - Potential form of know
\item ない/ません - Can't know (don't have the potential to know)
\end{itemize}
\begin{quote}
かもしれません - As to whether that is true or not I can can't go as far as to know
\end{quote}
\section{Lesson 23: だって, だから, それ から}
\label{sec:org0bada71}
だって is usually translated as 'because' and 'but' and 'even' and 'somebody said'. The reason for these myriad definitions is that だって isn't really a word.

だって is simply the copula だ, and って. This is not the て-form of だ, it is the same って as the contraction of the quotation particle という.

Firstly, \emph{somebody said}:
\begin{quote}
\textbf{(∅が) 明日は} (∅が) 晴れだ \textbf{って(という)} - \textbf{(Someone/weather forecast/people)} as for tomorrow (it) fine will-be \textbf{says}
\end{quote}

---

\emph{But:}
\begin{quote}
\begin{enumerate}
\item さくらがきれいだね - Sakura's pretty isn't she
\item だって 頭が 弱い - But she's not very smary (lit: head is weak)
\end{enumerate}
\end{quote}
Note: This usage usually has a childish, or somewhat argumentative tone.

Why does this mean 'but'? What're we're doing is taking the thing that was just said and adding だ to it. Then quoting what they just said: だって - You say that (Sakura is pretty), and the implication is that something contradictory will follow.

---

\begin{quote}
(\ldots{})だから - From that (what you/I just said) - Therefore; as above.
\end{quote}


---

\emph{Because:} Just as we can use だって as 'You say a thing is so, but\ldots{}' We can also use だって as 'You say a thing is so, this is because (some explanation)'.

What both of these phrases are saying is 'You have said something, and I don't dispute it, but here's something we can add to it that undermines the narrative that you're trying to put forward'. This is the same in both cases, it's only the translation to English that mandates a distinction.

---

\emph{Even:}
When we say だって to mean even we're not using だ in the same was as だから or in the other sense we've just spoken about, i.e. we aren't using it to refer back to the previous statement. We are usually attaching it to something in particular within the statement we're making.
\begin{quote}
\begin{enumerate}
\item さくらができる - Sakura can do that
\item わたしだって - Say (it) is me - I can do that
\end{enumerate}
\end{quote}
This has a different implication to わたしもできる which just means 'I can do that too', わたしだって, because it is associated with the phrases above carries the \emph{tone} of '\emph{even} I can do that'.

\begin{quote}
わたしだってホトケーキがつくられる - Even I can make hot cakes
\end{quote}
In this example we aren't saying anything contradictory to someone else but it still has the implication of \emph{even}.
\section{Lesson 24: Hearsay and guesses: \textasciitilde{}そうだ, \textasciitilde{}そうです}
\label{sec:org401d00f}
\subsection{Likeness}
\label{sec:org70f8864}
そう is a \emph{helper noun} that can mean either 'likeness', or 'hearsay'.

そう is the same そう as in the こう/そう/ああ/どう (in this way/in that way/in \emph{that} way/in what way) group we learned about in lesson 20. そう can be used with any of the three engines: う (verbs), い (adjectives) or だ (nouns). Simply remove the final kana and add そう for 'seems like' meaning. In the case of だ the engine must be an adjectival noun (な-adjective).

Remember that each of the three engines can be moved behind other cars to turn them into adjectives.

Once そう has been attached to an engine, the engine becomes a new adjectival noun.

\begin{quote}
元気だ -> 元気そう - Is healthy -> Seems healthy
\end{quote}

\begin{quote}
元気な学生 -> 元気そう学生 - Healthy student -> Healthy looking student
\end{quote}

\noindent\rule{\textwidth}{0.5pt}

\begin{quote}
おもしろい -> おもしろそうだ - Is interesting -> Seems interesting
\end{quote}
\begin{itemize}
\item Note: Logically だ must always be used with そう, but colloquially it is often left off.
\end{itemize}

Remember, in Japanese we can only say things that we actually know for ourselves, so unless we have read/tasted/experienced/whatever that thing which the other person is describing, we must say そう as we cannot \textbf{know} that is is the way that they say.

\noindent\rule{\textwidth}{0.5pt}

For verbs, in the case of ichidan we just remove る as usual, and in the case of godan we use the い-stem of the verb. The い-stem of a verb is what we might call the 'pure-stem' of a verb. In Japanese this is called 「連用形」(れんようけい) which means 'connective-use form'.

\begin{quote}
泣きそうだ - Seems about to cry
\end{quote}
\begin{itemize}
\item Note: Again; logically だ must always be used with そう, but colloquially it is often left off.
\end{itemize}

\subsection{Hearsay}
\label{sec:org54f8ed9}
The そう used for likeness is a \emph{suffix}, it is joined to other words in order to form a new word. Whatever the word was to start with, once そう is attached, it becomes an adjectival noun. This is not what happens with hearsay.

When talking about hearsay we use そうだ/そうです after the entire, complete sentence. Back to the train metaphor, the entire logical sentence becomes the main car at the core of a new sentence being pulled by a だ-engine.

\begin{quote}
さくらが日本人だ -> さくらが日本人だそうだ - Sakura is Japanese -> Sakura is Japanese I hear
\end{quote}
So simply put そうだ at the end of any full statement.

\section{Lesson 25: らしい vs そうです}
\label{sec:org884efc6}
らしい is a \emph{helper adjective}. Adjectives that end in しい we can consider a sub-class of adjectives that on the whole express subjectivities. That is to say, they are adjectives not describing an inherent property of something, but a (possible) human perspective on it:
\begin{itemize}
\item かなしい - Sad
\item うれしい - Happy
\item むずかしい - Difficult
\item やさしい - Easy
\end{itemize}

Like そう, らしい can be attached either to an individual word or to a complete logical clause/sentence. We don't need to change anything about the word, just attach らしい to it.

As with そう, if we attach らしい to a single word, we are talking about our (subjective) impression of that action or state.

And again as with そう, if we attach らしい to a sentence we're indicating that the statement itself is subjective, i.e. a deduction/hearsay/conjecture.

There is a difference however:
\begin{quote}
あの動物はウサギだ*そうだ* - I heard that animal is a rabbit
\end{quote}
\begin{quote}
あの動物はウサギだ*らしい* - It seems that animal is a rabbit
\end{quote}
These two \emph{can} mean the same thing, but not always. そう is specific to 'I heard', whereas らしい means that from the available evidence, which \emph{could} be what somebody said, or could be something else, it \emph{seems} to be a rabbit.

When it comes to the difference between そう and らしい with respect to a single word, the main difference is that we can't apply そうだ to a regular noun. We can only apply it to an adjectival noun. らしい can be applied to any noun, adjectival or otherwise.

らしい has the ability to liken one thing to another:
\begin{quote}
あの動物はウサギだ*らしい* - That animal is rabbit like
\end{quote}
らしい is not-necessarily conjecturing that something is something else, we may merely be saying that it is like that thing.
\begin{quote}
男らしい男 - Manly man
\end{quote}
\begin{quote}
さくら先生は先生らしくない - Sakura-sensei is not like (does not have the qualities of) a teacher
\end{quote}
\begin{quote}
それはさくらしくない - That wasn't like (you) Sakura
\end{quote}

\subsection{っぽい}
\label{sec:orgdbbe644}
\begin{quote}
こどもっぽい - Childish
\end{quote}
っぽい works very much like らしい and is also a \emph{helper adjective} but is much more casual than らしい.

っぽい cannot be used on the end of a completed clause, it can only be attached to a word.

らしい tends to imply that the quality is something that the thing ought to have, っぽい often tends to imply the opposite. This is not an absolute rule, just a tendency.
\begin{quote}
こどもらしい - Childlike
\end{quote}
\begin{quote}
こどもっぽい - Childish
\end{quote}
\section{Lesson 26: Similes - のように、 のような、 みたい}
\label{sec:org5a77c1d}
「ようだ」 forms the far end of a sliding scale of 'likeness' expressions:
\begin{center}
\begin{tabular}{lll}
Objectivity &  & Subjectivity\\
そうだ & らしい & ようだ + みたい\\
Hearsay/conjecture & Observation/quality comparison & Pure simile\\
\end{tabular}
\end{center}

Each of these expressions can be placed at the end of a complete logical sentence to express that the sentence is either something we've heard or some conjecture from the information available.

When we attach them to individual words then we have this sliding scale of meaning.
\begin{itemize}
\item そうだ - Conjecture of what the quality of something is: 「おいしいそうだ」 - It looks delicious
\item らしい - Has a much greater degree of subjectivity. 「らしい」 overlaps with 「そうだ」 in many respects but it can also do things that 「そうだ」 cannot. 「らしい」 can compare things to other things that we know they aren't: 「ウサギらしい」 - Rabbit-like (even though we know it isn't a rabbit). 「こどもらしい」 - Childlike (whether they are a child or not)
\item ようだ - 「ようだ」 can be much more subjective still, going as far as a \emph{metaphor} or \emph{simile}
\end{itemize}

\begin{quote}
山のようだ - Like a mountain
\end{quote}
\begin{quote}
風のように走る - Runs like the wind
\end{quote}

Often when 「ようだ」 is being used as a metaphor/simile it is used alongside 「まるで」:
\begin{quote}
日の丸 - Circle of the sun
\end{quote}
\begin{itemize}
\item まるで (usually does not use まる kanji) - roundly/wholly
\end{itemize}

\begin{quote}
まるで風のように走た - Wholly like the wind ran
\end{quote}
\begin{itemize}
\item Note: This is hyperbole. Hyperbole in this fashion is common in many languages. In English: 'I \emph{literally} froze to death'. We really mean: 'I \emph{figuratively} froze to death'; The hyperbole is to give emphasis.
\end{itemize}

We \textbf{cannot} use 「まるで」 with 「そうだ」 and we \textbf{shouldn't} use it with 「らしい」.

\begin{itemize}
\item のよう*だ* - Clause-end adjectival
\item のよう*な* - Pre-noun adjectival
\item のよう*に* - Adverb
\end{itemize}

「ようだ」 has a special usage that the other likening phrases don't have. As with the other two we can attach it to the end of a sentence with the meaning of 'seems/appears'. We can also attach it to a complete sentence for another purpose. We can use 「ようだ」 to turn an entire sentence into a simile:
\begin{quote}
まるで[1]ゆうれいを見た[2]か[3]のような[4]顔をした[5] - [5]Did (had) a face [1]exactly [4]as [3]if [2](she) saw a ghost - She made a face exactly as if she'd seen a ghost
\end{quote}
\begin{itemize}
\item Note: [2] is an entire sentence being used as a simile for (her) face.
\item Note: か is an as-yet uncovered particle roughly meaning 'if'. か bundles a statement into a kind of question.
\end{itemize}

We start with 「まるで」 indicating that we are going to use a simile. Then we make our completed statement. か turns our statement into a question, it gives us our 'if', she hasn't actually seen a ghost, it's a potentiality, so it's an 'if'. 「のように」 takes this and turns it into a simile 'as' i.e. 'as strong as\ldots{}', 'as if\ldots{}'. Finally we are describing her face we use the な connective form.

Using か in this way is something we can't do this with any of the other likening phrases. Even 「みたい」 which can do most of the things that 「のよう」 can do, can't do this.

「みたい」 is the less formal cousin of 「よう」 and means 'looks like\ldots{}', 'looks' not necessarily referring to solely literal vision:
\begin{quote}
山みたいだ
\end{quote}
\begin{quote}
山みたいな人
\end{quote}
\begin{quote}
風みたいに走る
\end{quote}
「みたい」 can also make true similes using 「まるで」 just like 「ように」. The main things to remember about 「みたい」 vs. 「ようだ」 is that 「みたい」 is less formal, and that we \textbf{can't} use it with a completed sentence. We \textbf{can} use it with a complete sentence for \emph{conjecture}, but not to create a \emph{simile}.

Because 「みたい」 is so casual, often the だ or 「です」 gets left off the sentence.
\section{Lesson 27: Bakari - ばかり}
\label{sec:org237589b}
「ばかり」 is essentially a noun, it is used だ and as we know we can only use だ with nouns, although we will find that sometimes the だ/です is left off in casual speech.

「ばかり」 simply means 'just' or 'nothing but'. One of the most common uses is to place it after a past-tense statement to say that something has 'just' taken place. This is just the same as in English.
\begin{quote}
来たばかり(だ) - I just came
\end{quote}
Why do we use the word 'just' in this way (in English and in Japanese)? It's another case of \emph{hyperbole}, just like in the last lesson. When we say 'just' we mean 'nothing else can have happened in so short a time', obviously this isn't literally true.

We can only attach 「ばかり」 to past-tense actions because in order for something to have 'just' happened it must have well, happened.

The next use of 「ばかり」 expresses that there is a great deal of something. Is this 'opposite' use just random? Once again we do the same thing in English.

We can say that someone or something does 'nothing but' an action by adding 「ばかり」 to the て-form of the word.
\begin{quote}
泣い[1]て[2]ばかり[3] - do[2] nothing but[3] cry[1]
\end{quote}
This example could be literal. It could also be exclusively figurative:
\begin{quote}
ゴルフをしてばかり - Only plays golf - Just plays golf (In reality they play an awful lot but not \emph{only})
\end{quote}

A conjunctions is something that connect two complete logical clauses in a compound sentence.
\begin{quote}
∅がうたった - Sentence 1
\end{quote}
\begin{quote}
∅がおどった - Sentence 2
\end{quote}
\begin{quote}
∅がうたったばかりかおどった - (A) not only [Sentence 1] but also [Sentence 2]
\end{quote}
There is a sense of unexpected or impressive positive or negative cumulation here.

The only only extra thing to understand here is the use of the か-particle. As covered in the last lesson か turns the statement it is attached to into an hypothesis or question. か can also, and especially in colloquial usage turn things into a negative. This is also done in English:
\begin{quote}
Do you think I'm going to do that? - I'm not going to do that!
\end{quote}
This is the same in Japanese. Sometimes we put か after something to say that it isn't the case.
\begin{quote}
∅がうたったばかりかおどった - (They) sing just don't, (also) dance. - They don't just sing (but also) dance. - Not only do they sing, they also dance.
\end{quote}

The other common conjunction made with 「ばかり」 is 「ばらいに」. に itself can something be added to form a conjunction, like with 「のに」. 「ばかりに」 is an \emph{explanatory} conjunction. It says 'something happened because\ldots{}'. The most common explanatory conjunctions are 「から」 and 「ので」 but 「ばかりに」 has a special implication, it's not simply saying that one thing happened because of something, it's saying that something happened \textbf{just} because of something.

\begin{quote}
みみか大きいばかりに誰もあそんでくれない - Just because my ears are big no one will play with me
\end{quote}
\section{Extra: のに and なのに}
\label{sec:org099d597}
「のに」 and 「なのに」 are often used as a sentence ending as well as a conjunction.

「のに」 is a conjunction, something that joins two clauses, each of which could be a complete sentence in of themselves, to form one compound sentence. 「のに」 is an \emph{opposing conjunction}, the main \emph{opposing conjunction} in English is 'but'. We use an opposing conjunction when the second clause stands in opposition to the first clause. The other opposing conjunctions in Japanese are 「けど」 and 「が」.
\begin{quote}
お店に行った*が/けど*パンがなかった - I went to the shops but there was no bread.
\end{quote}

「のに」 is a bit different to these two as it tends to (but doesn't always) imply that the second clause is unsatisfactory, and personally disappointing to the speaker.
\begin{quote}
パンがおいしいのにお客さんが來ない - Even though the bread is delicious, the customers don't come.
\end{quote}

How does this conjunction become a sentence ender? Grammatically speaking it doesn't. 「のに」 is still always a conjunction, but sometimes the second clause isn't given. These are called \emph{trailing statements}, statements that leave the conclusion unstated (but implied). These are very common in casual speech. Similarly we may often see sentences end with the て-form, grammatically this is wrong as the て-form is also a conjunction, but it's often done nonetheless.
\begin{quote}
赤ちゃんがうるさくて - The baby is/was noisy and so\ldots{}
\end{quote}
What this means exactly depends on context, it may mean that we couldn't sleep, or it was embarrassing.

\begin{quote}
さくらが約束したのに - Even though Sakura promised\ldots{} (she hasn't done something)
\end{quote}

What about 「なのに」? The reason sentences end in 「なのに」is that when we use の we have to use the conjunctive form of だ or 「です」. So when that first clause ends in だ, we have to change it to な in order to put a の after it.
\begin{quote}
∅が晴れた日曜日*だ* - It is a sunny Sunday (lit: it is a sunday that became sunny)
\end{quote}
\begin{quote}
∅が晴れた日曜日*な*のに - It is a sunny Sunday but\ldots{} (I can't go outside etc.) - Even though it is a sunny Sunday (I need to finish my homework etc.)
\end{quote}
\section{Lesson 28: ようになる, ようにする}
\label{sec:org19a4656}
When we use a noun followed by 「になる」 we mean that something turns into that noun:
\begin{quote}
さくらは∅がかえるになった - Sakura became a from
\end{quote}

「よう」 indicates a likeness to something:
\begin{quote}
山のようだ - Like a mountain
\end{quote}

When we use 「ようになる」 and the other phrases we don't add them to a noun, but to a complete logical clause:
\begin{quote}
(∅が)[かれを信じる]ようになった - (I) came to [believe him] - I became the state of believing him
\end{quote}

This is often used with the potential helper verb:
\begin{quote}
日本語のマンガが読*める*ようになった - Japanese manga became readable to me
\end{quote}

In both cases the state of something is changing. (My) state of not believing became a state of believing, the manga's state of being unreadable became a state of being readable.

「ようにする」 is the \emph{other-move} version of the 「なる」 construction, to make something enter a state:
\begin{quote}
まじょがさくらをかえるにした - Witch sakura frog-into did - Witch \emph{turned} Sakura into a frog
\end{quote}
\begin{quote}
よく見えるようにする - Make (someone/something) look good
\end{quote}

「ようにする」 also has an extended use meaning 'make sure':
\begin{quote}
ドアにかぎをかけるようにしてください - Please make it so that (you) lock the door
\end{quote}
Related to this is saying something about something that oneself does regularly:
\begin{quote}
毎日歩くようにする - (I) try to make it so that (I) walk every day
\end{quote}
Unlike 「ことにする」 (which will be covered next lesson) 「ようにする」 leaves a little wiggle-room or doubt. It's not something guaranteed, but it is the intent, hence 'try to'.

「ように」 can also be used as a conjunction to show \emph{cause-and-effect}. This isn't a separate grammatical usage, it only seems that way from the English translation:
\begin{quote}
よく見えるように口べにをつける - Look better, in order that, apply lipstick - Apply lipstick in order to look better
\end{quote}
In 「よく見えるようにする」 we're saying to make someone look better without specifying the means, we're just using the catch-all verb 「する」. In the conjunction example, the second clause is simply replacing the catch-all 「する」 with the actual means by which the first clause was accomplished.

One final note is that sometimes ように can be seen on the end of a sentence:
\begin{quote}
日本に行けますように - I wish (someone) could go to Japan
\end{quote}
This is most typically seen with 「ます」 sentences, and particularly in prayers or petitions. This is a shortening of 「ようにする」 or 「ようにしてください」.
\section{Lesson 29: ことになる, ことにする}
\label{sec:org8f28224}
As we know 「こと」 refers to an \emph{abstract} thing i.e. a situation or circumstance.
\begin{quote}
(∅が)(∅を)けっこんすることにした - (It) became the thing of getting married
\end{quote}
What is 'it'? It is the same thing it might be in English, the situation/circumstance in which getting married is the thing.

We have to use 「こと」 here because we cannot attach the logical particle に (or any logical particle) to anything but a noun. So we use 「けっこんする」 as a modifier for 「こと」 in order to give a noun for the situation/circumstance of getting married.

What does this really mean?
\begin{quote}
(∅が)(∅を)けっこんすることにした - We decided to get married - We brought about the situation of getting married - We brought about the situation in which getting married was the thing
\end{quote}

\begin{quote}
(∅が)フランスで留学する(∅が)ことになった - (I) France study (it) thing-in turned into - It became the thing of studying in France - It came about that I studied in France
\end{quote}
Because 「ことにする」 is a deliberate act, it is taken in many cases to imply a deliberate decision. Notice however there is no actual decision being made by anyone. 「ことになる」 on the other hand implies that something came about without our control:
\begin{quote}
(∅が)(∅を)けっこんすることになた - (It) came about that we're getting married
\end{quote}

\begin{quote}
(∅が)たいへんあことになった - (It) became a terrible thing
\end{quote}
This time 「ことに」 is not being used on a logical clause, just a single word and so does not carry the implication of a decision being made anywhere, as there is no action to be decided on.

\section{Lesson 30: Japanese conditionals: と}
\label{sec:org0ebea0d}
A \emph{conditional} is a statement like 'if' or 'when'. In Japanese there are many conditionals, in this lesson we will just cover と. We've already covered と as the \emph{exclusive-and} particle. This is exactly the same と as is used for conditionals. It's not the same と as the quotation-particle, but knowing that it is the same と as the exclusive-and particle makes understanding the conditional grammar much clearer.

と is a particle, but it's not a logical particle. It's also not a non-logical particle. Remember that it is an \emph{a-logical} particle, meaning that it carries the meaning of the logical particle attached to the second of the two nouns it connects. In the case of conditionals however, と doesn't attach to a noun, but to a logical clause.
\begin{quote}
冬になる*と*寒くなる - When it becomes winter it becomes cold
\end{quote}
Why is this connected by と? We're saying that when/if something happens, there is only one result.

と can also be used in a hyperbolic fashion.
\begin{quote}
(∅が)それを食べる*と*病気になる - Eat that and (you'll) get sick
\end{quote}
It is possible that someone might not get sick, but this is a hyperbole, we're trying to say to someone that 'if you eat that you \textbf{will} get sick'. 'If you keep playing those games you \textbf{will} fail the exam'.

We can also use this と to indicate that something is necessary:
\begin{quote}
行かない*と*ダメ - If I don't go it will be bad
\end{quote}
\begin{quote}
勉強しないといけない - If I don't study it won't go - If I don't study it won't do - I must study
\end{quote}

We may even here the 'if' on it's own without the 'then':
\begin{quote}
逃げないと! - If we don't run\ldots{} (something bad will happen) - We must run! - Run!
\end{quote}

Because と is exclusive, it's a bit more absolute and also a bit more colloquial than other conditionals like ば (to be covered later).

\section{Lesson 31: ば, れば conditional helpers}
\label{sec:orgd31128f}
The と conditional's particular characteristic was its exclusiveness, in many cases we can use the conditionals interchangeably without much changing in terms of meaning, but each conditional does still have its own unique qualities.

The ば/れば conditional is a helper that attached to the え-stem of a verb:
\begin{center}
\begin{tabular}{lllll}
あ & い & \textbf{う} & \textbf{え} & お\\
か & き & \textbf{く} & \textbf{け} & こ\\
さ & し & \textbf{す} & \textbf{せ} & そ\\
た & ち & \textbf{つ} & \textbf{て} & と\\
な & に & \textbf{ぬ} & \textbf{ね} & の\\
ば & び & \textbf{ぶ} & \textbf{べ} & ぼ\\
ま & み & \textbf{む} & \textbf{め} & も\\
ら & り & \textbf{る} & \textbf{れ} & ろ\\
\end{tabular}
\end{center}

Godan:
\begin{itemize}
\item かう -> かえ*ば*
\item きく -> きけ*ば*
\item はなす -> はなせ*ば*
\item もつ ->  もて*ば*
\item しぬ ->  しね*ば*
\item とぶ ->  とべ*ば*
\item のむ ->  のめ*ば*
\item とる ->  とれ*ば*
\end{itemize}
Ichidan:
\begin{itemize}
\item たべる たべ*れば*
\end{itemize}
Irregular:
\begin{itemize}
\item する -> すれば
\item くる -> くれば
\end{itemize}

For adjectives remove the い and add the helper 「ければ」. 「ない」 -> 「なければ」.

The special characteristic of ば/れば is that is used for \emph{hypotheticals}. They always mean 'if' and can't ever mean 'when', because we never know for sure if the condition will take place. Consequently, if we use it about something that happened in the past it has to be something that didn't happen, because if it had happened we would be dealing with a hypothetical.

This hypothetical nature allows ば/れば to be used in many common Japanese expressions:
\begin{quote}
どうすればいい? - How if I act will be good? How act if good? <-> Good if act how?
\end{quote}
\begin{itemize}
\item Note: the most apt translation of する most of the time is really 'act', not 'do':
\end{itemize}
\begin{quote}
しずかにする - Act quietly
\end{quote}

\begin{quote}
かさを持ってくればよかった - If I had brought an umbrella, that would've been good - I should have brought an umbrella
\end{quote}

\begin{quote}
勉強しなければいけない - If I don't study it won't go/it won't do - I must/should study
\end{quote}
\begin{itemize}
\item Note: There isn't actually a word for 'must' in Japanese, so we always use this 'If I don't do x it will be bad' construction.
\end{itemize}

Because this is a long winded way of saying something it often gets cut down:
\begin{quote}
。。。しなければ
\end{quote}
\begin{quote}
。。。しなけば
\end{quote}
Even in very casual speech it is still often said in full to emphasise the 'must':
\begin{quote}
なぜいかなければいけない - Why must I come?
\end{quote}
\section{Lesson 32: たら, なら conditionals}
\label{sec:orgad7795f}
The 「たら/だら」 conditional is particularly easy to form, all we do is form a verb or an adjective into its 「た/だ」 past form and add ら. 「たら/だら」 is the only conditional that can be used about past events. Of course, if we are referring to something in the past it's not really a conditional. We aren't saying 'if' here, we're saying 'when'. We know that the condition has been fulfilled because it's already happened. What 「たら」 does is show that the event that happened in the past was unexpected or surprising. This is because, rather than using on of the more regular means of showing that one event followed another such as using the て-form or 「から」, we're using an \emph{if-type conditional}. We're stressing the fact that what did happen might well not have happened, and that the thing not having happened might've been more inline with what we'd have expected.
\begin{quote}
家に帰ったらさくらがいた - When I returned to the house, Sakura was there (surprisingly). (She doesn't even have a key, she must've climbed in through the window. Sakura does that kind of thing from time to time y'know)
\end{quote}

We can also use 「たら」 on future events. Doing so tends to add stress on what will happen if the condition is fulfilled, as opposed to 「れば」 which poses more stress onto whether or not the condition will be fulfilled, or even the fact that it wasn't fulfilled. Again, 「たら」 is more 'when' than 'if'.

Sometimes the forms 「ったら」 and 「ってば」 to indicate exasperation.
\begin{quote}
さくらったら
\end{quote}
\begin{quote}
さくらってば
\end{quote}
What this literally means is a contraction of 「といたら」 or 「といえば」, in other words: 'When you speak of Sakura' i.e. 'Oh you Sakura\ldots{}', 'When it's Sakura it's always like this\ldots{}'. It's not flattering, it is critical, but it's not very strongly so. It can be quite humorous or joking. 「ってば」 is more likely to express real exasperation and can be put onto more things than just a person's name.
\begin{quote}
もう言ったってば - I've already said that haven't I?\ldots{}
\end{quote}

「なら」 is an easy to use particle like helper. All we do is put なら after what we say, which turns it into a conditional. We can put it after nouns and complete logical clauses. If used after a noun we don't need to use the copula (だ) probably because the な of 「なら」 has its roots in the copula itself.

「なら」 can be used on present and future conditionals that aren't in any real doubt at all. For example, if Sakura is worried that something may not be possible to her, we might say:
\begin{quote}
さくらなら、できる - If it's Sakura, it will be possible
\end{quote}
Of course we know it's Sakura, we're talking to Sakura. So what we're really saying is 'Since it's Sakura, it will be possible'.
\begin{quote}
\begin{enumerate}
\item 駅はどこですか - Where is the station
\item 駅なら、 あそくです - If it's the station you're asking for, it's over there
\end{enumerate}
\end{quote}
Of course there isn't any real doubt that it's the station that we're asking for, so again, 'since'.

\section{Lesson 33: Japanese limiting terms: だけ, しか, ばかり, のみ}
\label{sec:org9a02b5a}
\subsection{だけ}
\label{sec:orgbafd556}
「だけ」 means 'limit'. Often this is translated as 'only' and in it's most basic form 'only' is what we'd say in English, however it's important to realise that in order to understand some of its other uses, it really means 'limit'.
\begin{quote}
1000 円だけ持っている - (I) hold limit-of 1000 yen - I have only 1000 yen
\end{quote}
\begin{itemize}
\item Note: Implication neutral
\end{itemize}
「だけ」 functions essentially as a noun. In the set uses given here its particle (in this case を) can be dropped. In many other uses it takes a particle like any other noun.

\begin{quote}
できるだけ勉強します - To the limit of the possibility I will study - I will study if I can/as much as I can
\end{quote}
At this point if we think of 「だけ」 as 'only' we will begin to get confused. 「できろだけ」 means 'to the limit of possibility'.

\begin{quote}
留学しただけあって英語はうまい - Because of the limit of the fact (I) studied abroad, (my) English is excellent.
\end{quote}
This 「あって」 is the connective (て-form) form of 「ある」. The because in the sentence being the て form which remember often implies the cause of the following effect. A common translation of 「だけあって」 is 'not for nothing' but what's actually being said here is 'precisely because and only because (I) studied abroad (my) English is excellent'.

\begin{quote}
安いだけあってすぐに壊れちゃった - Because of the limit of it being cheap it quickly broke - Precisely because it was cheap, it broke quickly
\end{quote}
「だけ」 is being used for its ability to limit something down to something precise. Out of all the possible reasons that it could've broken, it is the reason that it is cheap. We have limited the possibilities down to just one. Again, 'only'.
\begin{quote}
留学しただけあって英語はうまい - Only by studying abroad did I get that good at English
\end{quote}
\begin{quote}
留学しただけあって英語はうまい - Only something really cheap would break that quickly
\end{quote}

\subsection{しか}
\label{sec:org57ed7f3}
「しか」 is often confusing because people are given the impression that it means more or less the same thing as 「だけ」.
\begin{quote}
1000 円だけ持っいない - (I) don't hold more than 1000 ten
\end{quote}
\begin{itemize}
\item Note: Implication
\end{itemize}
「しか」 means 'more than', so long as we understand this it is very simple. 「しか」 is only ever used in negative sentences, there is always a 「ない」 or an 「ありません」 when we use 「しか」, so it ends up saying 'not more than'.

「だけ」 doesn't imply that 1000 yen is a lot or a little, it only says that that is what we have and nothing else. 「しか」 implies that the 1000 yen we hold is not enough. 'I don't have any more than 1000 yen' places emphasis on the 1000 yen and it being too little or that if someone wants more they aren't going to get it.

\begin{quote}
にげるしかない - There's noting more (we can do) than run - There's nothing for it but to run - There's no other action but to run
\end{quote}

\subsection{ばかり}
\label{sec:orgaa94a40}
\begin{quote}
あのお店はパンだけうる - The shop only sells bread
\end{quote}
\begin{quote}
あのお店はパンしかうらない - That shops doesn't sell anything but bread
\end{quote}
\begin{quote}
あのお店はパンばかりうる - That shop sells an awful lot of bread
\end{quote}
「ばかり」 and 「だけ」 have the same literally meaning (only/just) but as we learned in the lesson on ばかり, it's a hyperbole, we don't literally mean it only sells bread. Just that it sells bread far more than anything else.

\subsection{のみ}
\label{sec:org4c38bd7}
「のみ」 is just 「だけ」 in its simplest sense i.e. 'only'.
\begin{quote}
あのお店はパンのみうる - The shop only sells bread
\end{quote}
「のみ」 is typically used in polite or formal contexts, otherwise 「だけ」 is used.
\end{document}
