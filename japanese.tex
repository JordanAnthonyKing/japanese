% Created 2021-03-17 Wed 20:23
% Intended LaTeX compiler: pdflatex
\documentclass[11pt]{article}
\usepackage[utf8]{inputenc}
\usepackage[T1]{fontenc}
\usepackage{graphicx}
\usepackage{grffile}
\usepackage{longtable}
\usepackage{wrapfig}
\usepackage{rotating}
\usepackage[normalem]{ulem}
\usepackage{amsmath}
\usepackage{textcomp}
\usepackage{amssymb}
\usepackage{capt-of}
\usepackage{hyperref}
\author{Jordan Cross}
\date{\today}
\title{Organic Japanese lesson notes}
\hypersetup{
 pdfauthor={Jordan Cross},
 pdftitle={Organic Japanese lesson notes},
 pdfkeywords={},
 pdfsubject={},
 pdfcreator={Emacs 28.0.50 (Org mode 9.4)}, 
 pdflang={English}}
\begin{document}

\maketitle
\tableofcontents

My follow-along notes from the Organic Japanese course on Youtube: \url{https://www.youtube.com/playlist?list=PLg9uYxuZf8x\_A-vcqqyOFZu06WlhnypWj}
\section{Lesson 1: The core Japanese sentence}
\label{sec:org951b510}
Every Japanese sentence is fundamentally the same, they have the same core.

Every sentence is formed of two elements, (visualised as a train) they are the \emph{carriage} and \emph{engine}.

In all languages there are only two kinds of sentence:
\begin{itemize}
\item A is B: Sakura is Japanese
\item A does B: Sakura Walks
\end{itemize}

Sakura ga nihonjin da - Sakira is Japanese
Sakura ga aruku - Sakura walks

Going back to the train metaphor:
\begin{verbatim}
 /Carriage/      /Engine/
[Sakura *が*] [Nihonjin *だ*]
\end{verbatim}

There is a third form of the Japanese sentence core sentence:
\begin{verbatim}
/Carriage/  /Engine/
[Pen *が*] [aka *い*] = Pen is red
\end{verbatim}

This is similar, but not identical to an adjective. This will be discussed more later.

To recap, all of these sentences begin with the subject, they are connected with が, and the three \emph{engines} of the Japanese core sentence are:
\begin{verbatim}
- う - verb          - A does B
- だ - noun          - A is B
- い - /"adjective"/ - A is B
\end{verbatim}

\section{Lesson 2}
\label{sec:org5b8c5b3}
\subsection{The invisible が carriage}
\label{sec:org9b6aee3}
Last lesson we covered the が \emph{carriage}. Part of the reason so many people struggle with Japanese is that, although we can always see the \emph{engine} of the sentence, we cannot always see the が carriage. Remember, the core of the Japanese sentence is formed of the が carriage and an engine.

The closest English equivalent to this invisible が carriage is 'it'. Here is an example English sentence:
\begin{quote}
The ball rolled down the hill. When the ball got to the bottom the ball hit a sharp stone and the ball was punctured and all the air came out of the ball.
\end{quote}

This is not a sentence we would ever say in English, once we have established that we are talking about 'the ball' we would instead refer to the ball with it:
\begin{quote}
The ball rolled downm the hill. When it got to the bottom it hit a sharp stone and it was punctured and all the air came out of it.
\end{quote}

If we were to completely omit \emph{it} the sentence would still be easy to understand, we don't \emph{need} to use this it marker each time, but English grammar \textbf{demands} it. Japanese does not, hence the 'invisible' が carriage.

'It' by itself doesn't really mean anything, we know what it means from context. If a child comes downstairs in the middle of the night and says \emph{'Got really hungry'}, \emph{'Came for something to eat'} we understand that the child means \emph{'I got really hungry'} not \emph{'The dog got really hungry'}. In English this isn't a proper sentence, but in Japanese it is.

All of the little pronouns I/it/we/he/they can all be replaced by the invisible が carriage; the \textbf{∅ pronoun}. It is important to remember that the carriage \textbf{is} still there.

One might say: ドリーだ meaning \emph{'I am Dolly'}. The full sentence being: ∅がドリーだ.

By default 'I' is the default value of this \emph{∅ pronoun}. However, if someone were introducing their daughter and said ドリーだ we would understand from context that ∅ meant this/she.

If I say 土曜日だ \emph{'It is Saturday'} it is clear that ∅ means \emph{today}.

Each of these sentences are complete grammatical sentences with a subject marked by が and an engine, but in each of these cases the が carriage is just invisible. It \textbf{is} still there. This may seem to be arbitrary, or over-complicated but it saves a lot of grief later on to model sentences this way. Without this information as sentences become more complex they're going to seem increasingly vague and hard to understand.

\subsection{The を carriage}
\label{sec:orgbce6591}
This carriage is formed of a noun and the particle を. The を particle marks the object of the sentence. The thing that some verb (the engine) is being done to (が marks the thing doing the verb). It is not part of the core sentence which is always formed of the が carriage and an engine.
\begin{verbatim}
/carriage/ /carriage/ /engine/
[わたしが]  [ケーキを]  [たべる]
    I        cake       eat
\end{verbatim}

The core sentence here is 'I eat'. The extra carriage, the を carriage is telling us more about the engine. \emph{What} are we eating? We are eating cake.

Once again, we would often see this said as: ケーキをたべる. This is just another case with the invisible が carriage. We \textbf{cannot} have a sentence without a が, we \textbf{cannot} have a sentence without a doer.

When we are saying ケーキをたべる, what we are really saying is ∅がケーキをたべる. And the default value for ∅ is わたし: \emph{I}.

\section{Lesson 3}
\label{sec:orge0f4b10}
\subsection{は particle secrets}
\label{sec:orged849bf}
The は particle can never be a part of the core Japanese sentence. It is neither the carriage we are saying something about, nor the engine, what we are saying about it. It isn't a carriage \emph{outside} of the core sentence either like the を particle is. The は particle is not part of the logical structure of the sentence.

は is a non-logical particle. In our train metaphor the は particle is a \emph{flag}. It simply marks something as the topic of the sentence, but doesn't say anything about it.

An exact translation of the は particle would be 'As for x'. わたしは therefore means 'As for me', \textbf{not} 'I am' (わたしが).

A commonly mistranslated sentence is:
\begin{quote}
わたしは日本人だ - I am Japanese
\end{quote}

If we look back at our train however we can see that something is missing:
\begin{verbatim}
  /flag/     /engine/
[わたし*は*] [日本人*だ*]
\end{verbatim}

There is no が carriage. We don't know who the subject actually is. One may ask 'well why don't we just treat the は particle as if it is a carriage'. In this example it is obvious that the topic marked by は is the same as the subject marked by が, but there are many more cases where this is not true, leading to much confusion down the road. Let's look at a similar sentence. You are at a restaurant, the waitress is asking what you would like:
\begin{quote}
わたしはうなぎだ ー \sout{I am an eel}
\end{quote}

Treating は as 'I am' doesn't work. As we now know the default value of the ∅ pronoun is 'I', but in this context it's clear that we're talking instead about \emph{what} we want to eat. わたしはうなぎだ therefore means 'As for me, eel'.

\subsection{The に particle}
\label{sec:orgaa7ff43}
The に particle marks the target (indirect object) of an engine. Along with the が and を we have a sort of \emph{trio} of logical \emph{A does B} sentences.

\begin{itemize}
\item が tells us who does the doing
\item を tells us what it is done to
\item に tells us what the ultimate target of that doing
\end{itemize}

\begin{quote}
わたしがぼーるをなげる - I threw the ball
\end{quote}
The \textbf{core} sentence is 'I threw', and the extra carriage (を) tells us what we threw, the ball. We can add another carriage to tell us more about the engine:
\begin{quote}
わたしがぼーるをさくらになげる - I threw the ball at/to Sakura
\end{quote}
Sakura is the destination, the target. It is important to note here that the logical particles tell us what is happened. The order of the words doesn't really matter the way it does in English.
\begin{quote}
わたしにさくらがぼーるをなげる - Sakura threw the ball at/to me.
\end{quote}
\begin{quote}
ぼーるがわたしにさくらをなげる - The ball throws Sakura at me
\end{quote}
Obviously this final example doesn't make any sense (although we might want to say something non-sensical like this in a fantasy novel or something) but we can say whatever we like in Japanese so long as we use the right logical particles.

Now let's introduce は:
\begin{verbatim}
  /flag/   /carriage/ /carriage/ /carriage/ /engine/
[わたし*は*]  [∅*が*]  [さくら*に*] [ぼーる*を*] [なげる]
\end{verbatim}
As we know, even if the が carriage is invisible (or silent) this means 'As for me, (I) threw the ball at Sakura'. Now let's give the は \emph{flag} to the ball:
\begin{verbatim}
  /flag/   /carriage/  /carriage/ /carriage/ /engine/
[ぼーる*は*] [わたし*が*] [さくら*に*]  [∅*を*]    [なげる]
As for the ball, I threw *it* (the ball) at Sakura.
\end{verbatim}
Note, this time the を carriage has become invisible, because what we're throwing is now marked by the は particle, ∅ here has taken the value of 'it'. Even without は we might already know what 'it' was that was thrown from context. The important thing to understand here is that as we change the logical particles from one noun to another we change the meaning of the sentence, but when we change the non-logical particle は from one noun to another it makes no difference to the logic of the sentence. It may make some difference to the emphasis, but it makes no difference to who is doing what or what they're doing it to.
\end{document}
