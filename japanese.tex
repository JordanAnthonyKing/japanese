% Created 2021-04-12 Mon 22:03
% Intended LaTeX compiler: pdflatex
\documentclass[11pt]{article}
\usepackage[utf8]{inputenc}
\usepackage[T1]{fontenc}
\usepackage{graphicx}
\usepackage{grffile}
\usepackage{longtable}
\usepackage{wrapfig}
\usepackage{rotating}
\usepackage[normalem]{ulem}
\usepackage{amsmath}
\usepackage{textcomp}
\usepackage{amssymb}
\usepackage{capt-of}
\usepackage{hyperref}
\author{Jordan Cross}
\date{\today}
\title{}
\hypersetup{
 pdfauthor={Jordan Cross},
 pdftitle={},
 pdfkeywords={},
 pdfsubject={},
 pdfcreator={Emacs 28.0.50 (Org mode 9.5)}, 
 pdflang={English}}
\begin{document}

\tableofcontents

My follow-along notes from the Organic Japanese course on Youtube: \url{https://www.youtube.com/playlist?list=PLg9uYxuZf8x\_A-vcqqyOFZu06WlhnypWj}
\section{Lesson 1: The core Japanese sentence}
\label{sec:org60b2fdb}
Every Japanese sentence is fundamentally the same, they have the same core.

Every sentence is formed of two elements, (visualised as a train) they are the \emph{carriage} and \emph{engine}.

In all languages there are only two kinds of sentence:
\begin{itemize}
\item A is B: Sakura is Japanese
\item A does B: Sakura Walks
\end{itemize}

Sakura ga nihonjin da - Sakira is Japanese
Sakura ga aruku - Sakura walks

Going back to the train metaphor:
\begin{verbatim}
 /Carriage/      /Engine/
[Sakura *が*] [Nihonjin *だ*]
\end{verbatim}

There is a third form of the Japanese sentence core sentence:
\begin{verbatim}
/Carriage/  /Engine/
[Pen *が*] [aka *い*] = Pen is red
\end{verbatim}

This is similar, but not identical to an adjective. This will be discussed more later.

To recap, all of these sentences begin with the subject, they are connected with が, and the three \emph{engines} of the Japanese core sentence are:
\begin{itemize}
\item う - verb          - A does B
\item だ - noun          - A is B
\item い - \emph{``adjective''} - A is B
\end{itemize}

\section{Lesson 2}
\label{sec:org8f32da1}
\subsection{The invisible が carriage}
\label{sec:org94fa6e8}
Last lesson we covered the が \emph{carriage}. Part of the reason so many people struggle with Japanese is that, although we can always see the \emph{engine} of the sentence, we cannot always see the が carriage. Remember, the core of the Japanese sentence is formed of the が carriage and an engine.

The closest English equivalent to this invisible が carriage is 'it'. Here is an example English sentence:
\begin{quote}
The ball rolled down the hill. When the ball got to the bottom the ball hit a sharp stone and the ball was punctured and all the air came out of the ball.
\end{quote}

This is not a sentence we would ever say in English, once we have established that we are talking about 'the ball' we would instead refer to the ball with it:
\begin{quote}
The ball rolled downm the hill. When it got to the bottom it hit a sharp stone and it was punctured and all the air came out of it.
\end{quote}

If we were to completely omit \emph{it} the sentence would still be easy to understand, we don't \emph{need} to use this it marker each time, but English grammar \textbf{demands} it. Japanese does not, hence the 'invisible' が carriage.

'It' by itself doesn't really mean anything, we know what it means from context. If a child comes downstairs in the middle of the night and says \emph{'Got really hungry'}, \emph{'Came for something to eat'} we understand that the child means \emph{'I got really hungry'} not \emph{'The dog got really hungry'}. In English this isn't a proper sentence, but in Japanese it is.

All of the little pronouns I/it/we/he/they can all be replaced by the invisible が carriage; the \textbf{∅ pronoun}. It is important to remember that the carriage \textbf{is} still there.

One might say: ドリーだ meaning \emph{'I am Dolly'}. The full sentence being: ∅がドリーだ.

By default 'I' is the default value of this \emph{∅ pronoun}. However, if someone were introducing their daughter and said ドリーだ we would understand from context that ∅ meant this/she.

If I say 土曜日だ \emph{'It is Saturday'} it is clear that ∅ means \emph{today}.

Each of these sentences are complete grammatical sentences with a subject marked by が and an engine, but in each of these cases the が carriage is just invisible. It \textbf{is} still there. This may seem to be arbitrary, or over-complicated but it saves a lot of grief later on to model sentences this way. Without this information as sentences become more complex they're going to seem increasingly vague and hard to understand.

\subsection{The を carriage}
\label{sec:org28a9169}
This carriage is formed of a noun and the particle を. The を particle marks the object of the sentence. The thing that some verb (the engine) is being done to (が marks the thing doing the verb). It is not part of the core sentence which is always formed of the が carriage and an engine.
\begin{verbatim}
/carriage/ /carriage/ /engine/
[わたしが]  [ケーキを]  [たべる]
    I        cake       eat
\end{verbatim}

The core sentence here is 'I eat'. The extra carriage, the を carriage is telling us more about the engine. \emph{What} are we eating? We are eating cake.

Once again, we would often see this said as: ケーキをたべる. This is just another case with the invisible が carriage. We \textbf{cannot} have a sentence without a が, we \textbf{cannot} have a sentence without a doer.

When we are saying ケーキをたべる, what we are really saying is ∅がケーキをたべる. And the default value for ∅ is わたし: \emph{I}.

\section{Lesson 3}
\label{sec:org7fff555}
\subsection{は particle secrets}
\label{sec:org98891bd}
The は particle can never be a part of the core Japanese sentence. It is neither the carriage we are saying something about, nor the engine, what we are saying about it. It isn't a carriage \emph{outside} of the core sentence either like the を particle is. The は particle is not part of the logical structure of the sentence.

は is a non-logical particle. In our train metaphor the は particle is a \emph{flag}. It simply marks something as the topic of the sentence, but doesn't say anything about it.

An exact translation of the は particle would be 'As for x'. わたしは therefore means 'As for me', \textbf{not} 'I am' (わたしが).

A commonly mistranslated sentence is:
\begin{quote}
わたしは日本人だ - I am Japanese
\end{quote}

If we look back at our train however we can see that something is missing:
\begin{verbatim}
  /flag/     /engine/
[わたし*は*] [日本人*だ*]
\end{verbatim}

There is no が carriage. We don't know who the subject actually is. One may ask 'well why don't we just treat the は particle as if it is a carriage'. In this example it is obvious that the topic marked by は is the same as the subject marked by が, but there are many more cases where this is not true, leading to much confusion down the road. Let's look at a similar sentence. You are at a restaurant, the waitress is asking what you would like:
\begin{quote}
わたしはうなぎだ ー \sout{I am an eel}
\end{quote}

Treating は as 'I am' doesn't work. As we now know the default value of the ∅ pronoun is 'I', but in this context it's clear that we're talking instead about \emph{what} we want to eat. わたしはうなぎだ therefore means 'As for me, eel'.

\subsection{The に particle}
\label{sec:orga38c065}
The に particle marks the target (indirect object) of an engine. Along with the が and を we have a sort of \emph{trio} of logical \emph{A does B} sentences.

\begin{itemize}
\item が tells us who does the doing
\item を tells us what it is done to
\item に tells us what the ultimate target of that doing
\end{itemize}

\begin{quote}
わたしがぼーるをなげる - I threw the ball
\end{quote}
The \textbf{core} sentence is 'I threw', and the extra carriage (を) tells us what we threw, the ball. We can add another carriage to tell us more about the engine:
\begin{quote}
わたしがぼーるをさくら*に*なげる - I threw the ball at/to Sakura
\end{quote}
Sakura is the destination, the target. It is important to note here that the logical particles tell us what is happened. The order of the words doesn't really matter the way it does in English.
\begin{quote}
わたし*に*さくらがぼーるをなげる - Sakura threw the ball at/to me.
\end{quote}
\begin{quote}
ぼーるがわたし*に*さくらをなげる - The ball throws Sakura at me
\end{quote}
Obviously this final example doesn't make any sense (although we might want to say something non-sensical like this in a fantasy novel or something) but we can say whatever we like in Japanese so long as we use the right logical particles.

Now let's introduce は:
\begin{verbatim}
  /flag/   /carriage/  /carriage//carriage/ /engine/
[わたし*は*]  [∅*が*]   [さくら*に*] [ぼーる*を*] [なげる]
\end{verbatim}
As we know, even if the が carriage is invisible (or silent) this means 'As for me, (I) threw the ball at Sakura'. Now let's give the は \emph{flag} to the ball:
\begin{verbatim}
  /flag/      /carriage/ /carriage/   /carriage/ /engine/
[ぼーる*は*] [わたし*が*] [さくら*に*]  [∅*を*]    [なげる]
As for the ball, I threw *it* (the ball) at Sakura.
\end{verbatim}
Note, this time the を carriage has become invisible, because what we're throwing is now marked by the は particle, ∅ here has taken the value of 'it'. Even without は we might already know what 'it' was that was thrown from context. The important thing to understand here is that as we change the logical particles from one noun to another we change the meaning of the sentence, but when we change the non-logical particle は from one noun to another it makes no difference to the logic of the sentence. It may make some difference to the emphasis, but it makes no difference to who is doing what or what they're doing it to.

\section{Lesson 4: Japanese past, present and future tenses}
\label{sec:org7b38c02}
Up until now we've only been using one tense and that is the one presented by the plain dictionary form of verbs. To use natural sounding Japanese we need 3 tenses. In Japanese these are not the same past, present and future tenses we're familiar with from English.

The tense we have been using thus far is \textbf{not} the present tense. It is the \emph{non-past} tense. This non-past tense is actually very similar to the English non-past tense. What is the \emph{English} non-past tense? It is again the plain dictionary form of a verb. Eat, run, walk etc. It is unnatural in English to say 'I eat cake', to mean 'I am eating cake'. It is natural however to use the non-past tense to say 'Sometimes I eat cake' or, in the explictily future tense 'I will eat cake'. Japanese is just the same as English in this way. It is rare we use this form for talking about things actually happening right now, except in cases like literary descriptions.

Most of the time the Japanese non-past tense refers to future events. In fact, just as ∅ defaults to 'I', the non-past tense defaults to the future.
\begin{quote}
さくらが歩く - Sakura will walk
\end{quote}
\begin{quote}
犬がたべる - dog will eat
\end{quote}
The way we have been using this tense up until now, 'Sakura walks', is possible, but isn't the most natural way.

If we want to say something more natural like 'Sakura is walking' we must use the verb 'to be'\footnote{In English the verb 'to be' is irregular and has multiple forms be/is/are/am: To \emph{be} walking, Sakura \emph{is} walking, not Sakura \emph{be} walking.}, or in Japanese いる.
\begin{quote}
さくらが歩いている - Sakura is walking
\end{quote}
\begin{quote}
犬がたべている - dog is eating
\end{quote}
There is something here however that we haven't yet seen. In our train metaphor this is a secondary engine, here たべて which could be an engine in of itself, is helping (modifying) the main いる engine. Our core sentence is still the same, we have a が carriage and an engine, いる i.e. さくらがいる - Sakura is (existing). The secondary engine modifies いる telling us more about what state she is currently existing in, she is in the eating state. As we go further into Japanese we will see this secondary engine structure again and again.
\begin{verbatim}
/carriage/ /engine/ /engine/
 [犬*が*]   [たべて]  [いる] - dog is eating
\end{verbatim}

Also, just as in English we don't say 'The dog is eat', we use a special form of the verb \emph{eat} => \emph{eating}. In Japanese this is the て form. This is covered in the next lesson.

For the past tense of verbs instead of adding て to verbs we add た.
\begin{quote}
犬がたべた - The dog ate
\end{quote}
The way in which we do this is exactly the same as the way in which we attach て and will be covered in the next lesson.

If we want to make it clear that we are talking about a future event we can add a time expression. By prefixing a sentence with あした (tomorrow), we can make it clear that what we will be doing, we will be doing tomorrow.
\begin{quote}
あした∅がケーキをたべる - Tomorrow I will eat cake
\end{quote}
Note how we simply preface the sentence with 'tomorrow', just like we would in English. This is the case with all \emph{relative-time nouns}, yesterday, tomorrow, the day after tomorrow, next week, next month, next year.

For \emph{non-relative}, i.e. \emph{absolute} time expressions we must use the に particle:
\begin{quote}
火曜日*に*∅がケーキをたべる - On Tuesday I will eat cake
\end{quote}
We must attach に in all the same places we would attach on/in/at in English. On Tuesday, in March, at 12 o'clock. Foruntately in Japanese we only need to use the one particle.

\section{Lesson 5: Japanese verb groups and て form}
\label{sec:orge49100d}
Japanese verbs fall into three groups: \emph{Ichidan}, \emph{Godan}, and \emph{irregular}

The first group are \emph{ichidan} (lit: one level) verbs. Morphing these verbs is easy, we simply remove the る and add our new ending. Ichidan verbs can only end in either いる or える.

The second group is by far the largest, the \emph{godan} (lit: five level) verbs. This groups contains verbs that end in all of the possible verb endings: う つ る ー ぬ ぶ む ー く ぐ ー す. Each of these ending groups has its own way of being morphed, though although they're 'five level' verbs, two of the groups use the same method so we only need to learn 4 methods. Confusingly this means that godan verbs can end in いる or える, most of these will still be \emph{ichidan} verbs, and fortunately even if a verb is morphed incorrectly, you will probably still be understood.
\begin{itemize}
\item う つ る -> って
\item ぬ ぶ む -> んで
\item く/ぐ -> いて/いで (Note: this is the combined group)
\item す -> して
\end{itemize}

There are only two irregular verbs, くる and する. いく, is partly irregular, but not completely.
\begin{itemize}
\item くる -> きて
\item する -> して
\item いく -> いって (\sout{いいて})
\end{itemize}
These are the only exceptions
\section{Lesson 6: Japanese ``adjectives''}
\label{sec:org742111b}
Japanese adjectives are not the same as English adjectives. As we have learned Japanese sentences come in three kinds, depending on the type of engine they have. As a reminder they are:
\begin{itemize}
\item う - verb - A does B
\item だ - noun - A is B
\item い - ``adjective'' - A is B
\end{itemize}

The truth is that all three of these types of engines can be used like adjectives.

Let's start with the first one, the one we refer to as an adjective in English, the い engine:
\begin{quote}
ぺんがあかい - Pen is red
\end{quote}
An important note, あかい does not mean 'red', it means 'is red'. あか means red.

If we swap the order of ぺんが and あかい then we can take this い engine, and now use it not as the primary engine, but as a secondary engine. This would not be a complete sentence however without a new engine, for example, a new (primary) い engine.
\begin{quote}
あかいぺんがちいさい - Red pen is small
\end{quote}
This is simple enough, let's take a look at verbs.

Any う (verb) engine, in any tense can be used like an adjective:
\begin{quote}
しょうじょがうたった - Girl sang
\end{quote}
\begin{quote}
うたったしょうじょが - The girl who sang (Note: this sentence is not yet complete, it lacks a primary engine).
\end{quote}
\begin{quote}
うたったしょうじょがねている - The girl who sang is sleeping
\end{quote}

Next, the noun engine:
\begin{quote}
いぬがやんちゃだ - The dog is naughty
\end{quote}
We can turn やんちゃ into an adjective too, but there is one important thing to note. Just as we have to add だ to a noun, here we must add な to the noun. な is the connective form of だ. Don't be fooled by 'な-adjectives', they're simply nouns!
\begin{quote}
やんちゃないぬが - The dog who is naughty (Note: this sentence is not yet complete, it lacks a primary engine).
\end{quote}
\begin{quote}
やんちゃないぬがねている - The dog who is naughty is sleeping
\end{quote}

An important note is that we cannot do this with \emph{all} nouns, only nouns which are frequently used in an adjectival way. This group of nouns is what the are referred to as 'な-adjectives'. We can use all nouns as adjectives, but for the rest we need to use a different technique and for that we will have to learn about the の particle.

\subsection{The の carriage}
\label{sec:org8e941c6}
The の particle, or the \emph{possessive particle} functions just like the English \emph{'s}.
\begin{quote}
さくら*の*はな - Sakura's nose
\end{quote}
\begin{quote}
わたし*の*はな - Me's (my) nose
\end{quote}
Luckly in Japanese we don't have to worry about his/her/my/their, we just use の.

Because this is the \emph{possessive particle} we can use this in another slightly different way. あか has an \emph{adjectival} form in あかい, but not all colours have this form. The Japanese for pink, ピンクいる (lit: pink-colour) doesn't have an adjectival form in ピンクいろい, nor can we use it as a secondary engine with な. So what are we to do? Well we can use the の particle:
\begin{quote}
ピンクいる*の*どれすが - The pink dress (literally: The dress belonging to the class of pink things)
\end{quote}
\begin{quote}
うさぎ*の*OSCAR - Oscar the rabbit (literally: Oscar belonging to the class of rabbit)
\end{quote}

Just as before, there's no need to worry about misusing の and な, no-one listening is going to misunderstand what you're saying and it's a very typical beginner mistake to make.

Using these techniques we can make all kinds of sentences that can become very complex, especially with verbal adjectives in which we can use whole sentences in an adjectival manner.
\section{Lesson 7: Negative verbs and adjective ``conjugations''}
\label{sec:orgefac20d}
The fundamental basis of negatives is the adjective ない. This adjective means 'non-exist'. The word for exist (for any inanimate thing) is ある. If we want to say that something exists:
\begin{quote}
ぺんがある
\end{quote}
But if we want to say that something doesn't exist we say:
\begin{quote}
ぺんがない
\end{quote}
Now, why do we use a verb for being, and an adjective for non-being? This is something that happens all throughout Japanese, when we do something we use a verb, but when we don't do something we attach ない and are therefore using an adjective as the engine of the sentence. This has a very logical reason, when we \textbf{do} something, an action is taking place, and so we use a verb, but when we \textbf{don't do} it we are describing a state of non-action, so that's an adjective.

Above we have said 'There is no pen', how do we say 'This is not a pen'?
\begin{quote}
これはぺんがある - As for this, it's a pen
\end{quote}
\begin{quote}
これはぺんではない - As for this, as for being a pen, it's not. (Note: で is the て-form of だ)
\end{quote}

Now let's look at negative verbs. To make a verb negative we must attach ない to the あ-stem of the verb. How do we do this?

\subsection{The Japanese stem system}
\label{sec:org4800744}
Note: these stems apply to Godan verbs. For Ichidan verbs we simply drop the る and add ない. Remember, all ichidan verbs end in る but not all る ending verbs are ichidan verbs.
\begin{quote}
たべる ー> たべない
\end{quote}

Here is the kana-grid, presented on its side. Every verb ends in one of the う-row kana. (う-row kana that aren't used as verb endings have been removed).

\begin{center}
\begin{tabular}{lllll}
あ & い & \textbf{う} & え & お\\
か & き & \textbf{く} & け & こ\\
さ & し & \textbf{す} & せ & そ\\
た & ち & \textbf{つ} & て & と\\
な & に & \textbf{ぬ} & ね & の\\
ば & び & \textbf{ぶ} & べ & ぼ\\
ま & み & \textbf{む} & め & も\\
ら & り & \textbf{る} & れ & ろ\\
\end{tabular}
\end{center}

As we can see there are four other ways in which the verb could end. These are the verb stems. For now we're only looking at the あ-stem as this is the one we need for the negative.

To from the あ-stem we simply shift the final kana from the う-row to the あ-row. There is one only exception and this is the only exception in the entire stem system. This exception is that う itself does not become あ but わ. This is because, take for example a verb like かう, かあ would not be as easy to say as かわ. Every other う-row kana is simply changed to its あ-row equivalent.

\begin{center}
\begin{tabular}{lllll}
\uline{\textbf{わ}} & い & \textbf{う} & え & お\\
\uline{か} & き & \textbf{く} & け & こ\\
\uline{さ} & し & \textbf{す} & せ & そ\\
\uline{た} & ち & \textbf{つ} & て & と\\
\uline{な} & に & \textbf{ぬ} & ね & の\\
\uline{ば} & び & \textbf{ぶ} & べ & ぼ\\
\uline{ま} & み & \textbf{む} & め & も\\
\uline{ら} & り & \textbf{る} & れ & ろ\\
\end{tabular}
\end{center}

So to form the negative form of a verb convert it to the あ-stem and add ない.
\begin{quote}
かう ー> かわない
\end{quote}
\begin{quote}
はなす ー> はなさない
\end{quote}

\subsection{Negative adjectives (and adjective ``conjugations'')}
\label{sec:orgf5622d2}
The adjective stem is simple, just drop the い and add く. This is how we make the て form, あかい ー> あかくて, and it's also the way we make the negative, あかい ー> あかくない.

If we want to put an adjective into the past tense we drop the い and add かった.
\begin{quote}
こわい ー> こわかった - Was scary
\end{quote}

Because ない is also an adjective, the past tense of it is just なかった.

\begin{center}
\begin{tabular}{lll}
Non-past & Past & \\
\hline
さくらがはしる & さくらがはしった & Positive\\
さくらがはし*ら*ない & さくらがはし*ら*なかった & Negative\\
\end{tabular}
\end{center}

Now as we know さくらがはしる is not very natural Japanese, instead we would say さくらがはしっている. For this, all we need to do is put the いる into the past tense:
\begin{quote}
さくらがはしっている -> さくらがはしっていた - Sakura was running
\end{quote}

\subsection{The only exceptions}
\label{sec:orgc7e6ba7}
There are only two real exceptions to what has been covered in this lesson. They are the helper verb ます which makes words formal by adding it to the い-stem of a verb. When we put ます into the negative it does not become まさない as we would expect, but becomes ません, because it is formal it is a bit old-fashioned and uses the old Japanese negative せん instead of ない.

The only other exception is いい (is good), which has an older form, which is still widely used in よい. When we morph いい it becomes よい again:
\begin{quote}
いい ー> よくない - Not-good
\end{quote}
\begin{quote}
いい ー> よかった - Was-good
\end{quote}

Note: よかった is a common phrase: ∅がよかった - \emph{It was good (That went well, it turned out great etc.)}
\section{Extra: The secret to all Japanese ``conjugations''}
\label{sec:orgc10ed36}
Uncovered. Partially covered in previous lesson. Will be covered if required in future lessons.
\section{Lesson 8: Location, purpose and transformation (に and へ particles).}
\label{sec:org5c0e5fe}
We already know that in a logical sentence the に particle marks the ultimate target of an action. If we are going somewhere, or sending something somewhere, or putting something somewhere, we use に.
\begin{quote}
∅がみせ*に*いく - (I) will go to the shop
\end{quote}
We can also mark a more subtler kind of target:
\begin{quote}
∅がみせ*に*たまごをかい*に*いく - (I) will go to the shop to buy eggs
\end{quote}
Note: かい is the い stem of かう, to buy.

If we recall, logical particles (が, に, を) can only mark nouns. The い stem of a verb is the equivalent noun form of it. Just as in English 'I like swimming', \emph{swimming} is a noun, 'I go to the shop for the purpose of buying eggs', this \emph{buying} is also a noun.

に gives us the target of an action in the literal sense, and also the target in a volitional sense, i.e. the aim of our action.

As well as identifying a place we will go to, に can also specify a place we are currently at:
\begin{quote}
∅がみせ*に*いる - (I am) at the shop
\end{quote}
This に is still marking a target, just not a future target. In order for something to be somewhere it must've gotten there, and so に specifies the target or some past action. We can also use this for inanimate objects:
\begin{quote}
ほんは∅テーベルのうえ*に*ある - As for the book, (it) exists (is) on-at the table.
\end{quote}
Note: うえ is a noun, meaning the on/top of something.

Finally, に can also mark a transformation. If \emph{a} becomes \emph{b}, then に also marks \emph{b}, the thing a is becomming.
\begin{quote}
さくらは∅がかえる*に*なった - Sakura became a frog
\end{quote}
Of course this example is a bit of a joke, but there are of course various every day things that become other things. This form of expression is also used much more often in Japanese than in English.
\begin{quote}
ことし∅が十八さいになる - This year (I) become 18 years old
\end{quote}
\begin{quote}
あとで∅がくもりになる - Later (it (the weather) will) become cloudy
\end{quote}

For adjectives things work slightly differently:
\begin{quote}
さくらがうつくしい - Sakura is beautiful
\end{quote}
If we want to say 'Sakura became beautiful' we can't use に because うつくしい isn't a noun, (referring back to our metaphor), it's not a carriage, it's an engine. All we need to do is turn the adjective into its stem by removing い and adding く (refer back to lesson 6).

\begin{quote}
さくらがうつくしくなった - Sakura became beautiful
\end{quote}

\subsection{The へ car}
\label{sec:org4313775}
Note: when used as a particle へ is pronounced え.

This is a very simple particle, it duplicates a single use of に. When we say \emph{a is going to b} we can freely substitute に with へ. This is \textbf{all} is can do, it cannot even mark the case where something \emph{is}, only where it is \emph{going}.
\section{Lesson 8b: Japanese particles explained}
\label{sec:org8fe47fc}
A logical particle tells us how the sentence logically holds together. It tells us who does what to whom with what, when and where.

は is a non-logical particle, it simply identifies the topic, but doesn't say anything about it. Other particles like と are \emph{alogical}, they aren't simply markers. In the case of と the particle 'ands' two nouns together. It is therefore doing something in the sentence, in our train metaphor it is joining a noun-carriage to another carriage, inheriting its logical particle, but has no function of its own.
\begin{quote}
さくら*と*メイリー*が*あるいていた - Sakura \textbf{and} Mary were walking
\end{quote}

Logical particles \textbf{always} attach to a noun. If we see a logical particle attached to anything else then we know that that word is functionally a noun.

The noun and the particle attached to it are an inseperable pair. We must view the two together, they are a question and an answer which form a fundamental unit of the Japanese sentence.

There cannot be a sentence without が, even if sometimes you can't see it. が can work in A is B sentences, descriptive sentences. The other particles can \textbf{only} work in A does B sentences, that is sentences with a verb engine.
\begin{itemize}
\item が - Who (or what) did it?
\item を - Whom was it done to?
\item に - Where did they go?/Where are they?
\item へ - What direction?
\item で - Where was it done? With what was it done?
\end{itemize}
\begin{quote}
∅がこうえん*に*いる - I am in the park
\end{quote}
\begin{quote}
∅がこうえん*で*あるんでいる - I am playing in the park
\end{quote}
Remember, あそんで is the て form of あそぶ and is a secondary engine, modifying いる. I am => I am playing.
\begin{quote}
∅がこうえん*に*いく - I go to the park
\end{quote}
\begin{quote}
∅がバス*で*こうえんいいく - I go to the park by bus
\end{quote}
If we say 'I went by bus' or 'I ate with chopsticks' we use で for the thing we did it with, the means by which we performed the action.
\section{Lesson 9: Japanese: No 1 secret and expressing desire}
\label{sec:org468d074}
\subsection{The No 1 secret}
\label{sec:org39ceeb5}
English is an \emph{ego-centric} language. Japanese is a more \emph{animist} language. What this means is that English always wants a person, preferably I, but if not I then someone else, or perhaps it will settle for an animal, but always wanting an animate being to be acting. Japanese is not this way:
\begin{quote}
わたしはコーヒがすきだ
\end{quote}
A textbook would translate this as 'I like coffee'. 'I like coffee' very well might be the English equivalent for this simple phrase, but it is not what this sentence means, and it is not what this structure of sentence means.

The が is marking the coffee. The coffee is the actor in this sentence, not I. It's not 'I like coffee', I am not \emph{liking} it. But 'As for me, coffee is likeable/pleasing'.

The English 'I like coffee' is an \emph{A does B} sentence. The Japanese is an \emph{A is B} sentence.

すき is a \textbf{noun}. An adjectival noun, but still a noun. It is not a verb like in English.

If this sentence were 1-1 with the common English meaning given, then every single part of it would be misdescribed by the particles.
\begin{itemize}
\item は does not mark an actor
\item が does not mark an object
\item だ does not mark a verb
\end{itemize}

\begin{quote}
わたしはほんがわかる
\end{quote}
\subsection{Expressing desire}
\label{sec:org4664857}
\begin{quote}
わたしは(optional)こねこがほしい - As for me, a kitten is wanted
\end{quote}
ほしい is often translated in English as 'want', but again, it is not a verb, it is an adjective. Again, \emph{I} is not the actor of the sentence, it is the cat, and it is the cat that is wanted, not 'I want a cat'.

In Japanese the way that wanting to \textbf{do} something is expressed is different to the way that wanting to \textbf{have} something is expressed.

The way this is done is with the い-stem again. To express that we want to do something, we must add the \emph{helper adjective} たい to the い-stem of the verb. たい doesn't mean 'want' in the English sense, it can't, again, because \textbf{want} is a verb, and たい is an \textbf{adjective}.
\begin{quote}
わたしは(optional)クレープガたべたい
\end{quote}
The common English translation for this is 'I want to eat crepes', but as we see the pattern is just the same as in the other cases, the desireability of the crepes is not a verb, it is an adjective.

There is no truly good translation of this into English. We shouldn't be thinking in terms of 'awkward English' or 'natural English' when it comes to constructing and understanding these sentences. We should be thinking in terms of Japanese. The 'awkward' translations of the Japanese are only there to give a \textbf{grasp} of the structure of Japanese.

Now, what if we took this sentence, わたしはクレープガたべたい and removed the optional parts so that we just had たべたい (∅がたべたい)? In this case, the meaning of the sentence would be what the common English translation is. ∅ defaults to I, and so the translation is 'I eat-wanting am' => 'I am wanting to eat' => 'I want to eat'. Because there is no eat-inducing subject here, the want to eat is attributed directly to I.

So what is たい? Is it an adjective describing the \emph{condition of something} making you want to do something, or is it an adjective describing \emph{my desire}? Well, it can be either. This is very common throughout Japanese. こわい can mean scared or scary.
\begin{quote}
おばけがこわい - Ghosts are scary
\end{quote}
\begin{quote}
∅がこわい - I am scared
\end{quote}
This isn't confusing because が tells us what to do.

\subsection{A final note to help keep things clear}
\label{sec:org437b2d6}
We cannot use these adjectives of desire (or any emotion) about anyone other than ourselves. If we say たべたい and there is no context to give the subject, then we must be talking about ourselves, and never the person we are speaking to or anyone else. Japanese simply doesn't allow us to use たい or こわい or ほしい or anything else about anyone other than ourselves.

If we wanted to say that someone else wants something then, because Japanese is such a logical language it doesn't allow us to say something that we cannot know for sure. One thing that we cannot know for sure is someone's inner feelings. We might think that Sakura wants to eat cake, but we can't know for sure. So if I want to talk about her desire to eat cake, we can't just use たい. We need to add to たい (or こわい, or ほしい or anything else) the helper verb がる.

To do this we take the い off of the adjective and add the helper verb がる.
\begin{itemize}
\item たがる
\item こわかる
\item ほしがる

がる means 'to show signs of', 'to look as if it's the case'.
\end{itemize}
\begin{quote}
さくらがケーキをほしがる - Sakura is showing signs of wanting cake
\end{quote}
Even if Sakura has actually told me she wants cake, we must still use がる. All I know is what she's said, I still don't know her feelings absolutely.

Why do we use a verb for other people and an adjective for ourselves? I can't describe someone else's feelings because I don't know about them, I can only describe their actions, and their actions are a verb.
\section{Lesson 10: Japanese ``conjugation'' and potential form}
\label{sec:org4977075}
What textbooks typically refer to as ``conjugations'' are really just helper-verbs attached to the four verb stems.

We have already looked at the helper-adjectives ない and たい, as well as the helper-verb がる. Now for the potential helper-verb which attaches to the え-stem of a verb.
\begin{center}
\begin{tabular}{lllll}
あ & い & \textbf{う} & \textbf{え} & お\\
か & き & \textbf{く} & \textbf{け} & こ\\
さ & し & \textbf{す} & \textbf{せ} & そ\\
た & ち & \textbf{つ} & \textbf{て} & と\\
な & に & \textbf{ぬ} & \textbf{ね} & の\\
ば & び & \textbf{ぶ} & \textbf{べ} & ぼ\\
ま & み & \textbf{む} & \textbf{め} & も\\
ら & り & \textbf{る} & \textbf{れ} & ろ\\
\end{tabular}
\end{center}

The potential helper verb has two forms, for godan verbs る, and for ichidan verbs られる.

There are only two exceptions, くる and する.
\begin{quote}
くる -> こられる
\end{quote}
\begin{quote}
する ー> できる
\end{quote}

There is only one difficult spot with the potential form, and it is similar to the issue from the previous lesson.
\begin{quote}
わたしは(optional)ほんがよめる - As for me, the book is readable
\end{quote}
A common translation of this would be 'I can read the book', however again the が is on the book, not on I. If we wanted to say 'I can read the book', the book would need to bemarked by を as it is the target of our reading, and 'I' would have to be marked by が as I is the actor.
\begin{quote}
わたしがほんをよめる - I can read the book
\end{quote}
This is perfectly fine, but it's not what is usually done. Remember, Japanese is not \emph{egocentric}.

As we're using a helper-verb, the past, non-past, negative-past and negative-non-past conjugation rules are the same as regular verbs, for あるける (can walk):
\begin{itemize}
\item あるける - non-past
\item あるけた - past
\item あるけない - negative-non-past
\item あるけなっかた - negative-past
\end{itemize}
\section{Lesson 11: Compound sentences, くれる, あげる, and more て form uses}
\label{sec:org5ee2687}
\begin{quote}
ある 日 アリスは 川の そばに いた。
\end{quote}
そば is a noun meaning 'beside', so 川のそば means beside the river.

ある means 'a certain', so ある日 means 'on a certain day'. Notice that here we're using a verb as an adjective, as we have described in previous lessons. 本がある (a book exists) -> ある本が (an existing/a certain book). Note also that this is the same as how we use might use Today, Yesterday, Tomorrow etc. but not how we might use 'On Saturday'.
\begin{quote}
On a certain day, alice was beside a river.
\end{quote}

\begin{quote}
おねえちゃんは つまらない 本を よんで いて あそんで くれなかった。
\end{quote}
よむ (read) -> よんでいる (reading) -> よんでい*て* - We have put the いる into the て form, why have we done this?
\begin{quote}
おねえちゃんは つまらない 本を よんで いる - Big sister is reading an uninteresting book.
\end{quote}
This by itself is a complete clause (sentence), by turning the engine of the sentence (いる) into the て form we're saying that something else is going to follow this clause, i.e. 'and'.
\begin{quote}
おねえちゃんは つまらない 本を よんで いて - Big sister is reading an uninteresting book and\ldots{}
\end{quote}
\begin{quote}
あそんで くれなかった
\end{quote}
あそぶ is to play. This has also been put into the て form. Here we have another use of the て form, let's examine. くれる means to 'give downwards' i.e. as Japanese is so polite we place ourselves below others, so someone else is giving us something. あげる to contrast is to 'give upwards' i.e. to give to someone else. What is being given? In this case the thing being given is what is attached to it via the て form, i.e. 'playing'. Specifically, she is not giving the act of playing to Alice. In Japanese we frequently use 'give' for actions, for doing something for our benefit as well as for literally 'giving' nouns. If someone does something for our benefit, we turn that action to the て form, and attach it to くれる. If we do something for someone else's benefit, we turn that action to the て form, and attach it to あげる.
\begin{quote}
Big sister is reading an uninteresting book and didn't play with Alice (didn't play for Alice's benefit).
\end{quote}

Note again our two clauses:
\begin{quote}
おねえちゃんは つまらない 本を よんで いて
\end{quote}
\begin{quote}
あそんで くれなかった
\end{quote}
For the first clause we do not know what in what tense the action is taking place. In English we would place the tense marker on both clauses, in Japanese we only do this at the end. よんで いて could mean 'is reading' and it could mean 'was reading'. Because くれなかった is in the past tense, then the entire sentence is in the past tense.
\section{Lesson 12: と quotation particle and compound verbs and compound nouns}
\label{sec:org64aad46}
\begin{quote}
「おもしろい ことが ない」 と アリスは 言った
\end{quote}
\begin{itemize}
\item もの - thing concrete
\item こと - thing abstract
\end{itemize}

\begin{quote}
「おもしろい ことが ない」- No interesting (abstract) thing exists (Nothing interesting is going on here)
\end{quote}
\begin{quote}
アリスは 言った - Alice said
\end{quote}
The intesting thing to note is the particle between these two, the と particle. There are two と particles, one means 'and', and one marks a quotation. When we quote someone as saying something or as thinking something we use this と particle. We also use these square brackets which are the equivalent of English quotation marks, but in speech we cannot see these, so we also use と (and clearly use と in writing regardless also).
\begin{quote}
「おもしろい ことが ない」 と アリスは 言った - Nothing interesting is happening said Alice
\end{quote}

\begin{quote}
そのとき、白い ウサギが とおり すぎた。
\end{quote}
\begin{itemize}
\item そのとき - That time, in this sentence it is used to mean 'just at that moment' (just as Alice said that)
\item とおる - pass through
\item すぎる - exceed, go beyond
\end{itemize}

とおりすぎる is doing something very interesting. It is attaching the い stem of one verb, to another verb to give it extra meaning. We will see this a lot throughout Japanese. Connecting とおる and すぎる, 'pass through' and 'go beyond' means 'passing by'.
\begin{quote}
そのとき、白い ウサギが とおり すぎた。 - At that moment a white rabbit passed by
\end{quote}

\begin{quote}
ふつうの ウサギでは なくて、 チョッキを きて いる ウサギ だった。
\end{quote}
\begin{quote}
ふつうの ウサギでは なくて - Not an ordinary rabbit
\end{quote}

\begin{quote}
チョッキを きて いる ウサギ だった。- It was a rabbit that was wearing a vest (it was a wearing a vest rabbit)
\end{quote}


\begin{quote}
ウサギは かいちゅうどけいを 見て 「おそい!おおい!」と言って、はしり だした。
\end{quote}
\begin{itemize}
\item かいちゅうどけい is the combination of かいちゅう (inside pocket) and どけい (watch) meaning pocketwatch. An interesting note here is that とけい becomes どけい (I may write up a full explanation of why anothe time).
\end{itemize}

\begin{quote}
「おそい!おおい!」と
\end{quote}
What と does structurally, is it takes whatever it marks which could be a whole paragraph, or just two short words like this, or anything with all sorts of grammar going on, it takes whatever it marks as a quotation and turns it into a single noun. Going forward we will find that this is used not only to mark things people say and people think, but to mark all sorts of things. This と structure can therefore make a quotation act as a modifier to whatever follows, in this cast it is modifying 言う (to say), or to think or feel, but could be many things.

Note: As there must be a ∅ somewhere the rabbit is either saying 'it's late' or 'I'm late'.
Note also, we don't need to use ウサギは again as it is already marked in the first half of the sentence.
\begin{quote}
はしり だした - Run + Take out = Broke into a run (started to run). In this sense だした means modifies the verb to mean the action 'erupted'.
\end{quote}
\begin{quote}
ウサギは かいちゅうどけいを 見て 「おそい!おおい!」と言って、はしり だした。 - The rabbit looked at his watch and said 'I'm late! I'm late' and broke out into a run.
\end{quote}
Note: all the uses of て-form to mark 'and'.
\begin{quote}
「ちょっと まって ください」 と アリスは よんだ。
\end{quote}
\begin{quote}
「ちょっと まって ください」 と - Please wait a little
\end{quote}
\begin{quote}
「ちょっと まって ください」 と アリスは よんだ。
\end{quote}
よんだ is the て-form of よぶ (to shout/call).
Note: Both よむ and よぶ conjugate to よんだ in the て form, fortunately it's not likely that we'll get these two verbs mixed up.
\begin{quote}
「ちょっと まって ください」 と アリスは よんだ。 - Please wait a minute called Alice
\end{quote}

\begin{quote}
でも ウサギは ピョンピョンと はしり つづけた。
\end{quote}
\begin{itemize}
\item でも - but
\item はしる + つづける = Continued running (running continued).
\item ピョンピョン - The sound of a small thing jumping along
\end{itemize}

Once again we're using the quotation particle と to describe the way in which it run, it ran in the way it sounds, it ran like a small thing jumping along (note there are no quotation marks around this).
\section{Lesson 13: Passive ``conjugation'' - Not passive and not a conjugation}
\label{sec:orga64ee1f}
The real name for the 'passive conjugation' is the \emph{Receptive helper verb}.

The receptive helper verb is れる for godan verbs and られる for ichidan verbs, and attaches to the あ-stem of another verb.
\begin{center}
\begin{tabular}{lllll}
\textbf{あ} & い & \textbf{う} & え & お\\
\textbf{か} & き & \textbf{く} & け & こ\\
\textbf{さ} & し & \textbf{す} & せ & そ\\
\textbf{た} & ち & \textbf{つ} & て & と\\
\textbf{な} & に & \textbf{ぬ} & ね & の\\
\textbf{ば} & び & \textbf{ぶ} & べ & ぼ\\
\textbf{ま} & み & \textbf{む} & め & も\\
\textbf{ら} & り & \textbf{る} & れ & ろ\\
\end{tabular}
\end{center}
Remember! う becomes わ, not あ.

The receptive helper verb means \emph{receive} or \emph{get}, we're receiving/getting the action that the helper verb is attached to
\begin{quote}
さくらがしか*ら*れた - Sakura scolded-got - Sakura got scolded/Sakura received a scolding
\end{quote}

Note, the receptive helper verb and the modified verb have different actors. The sentence is not Sakura scolds, someone else (we don't know who) is scolding Sakura, but Sakura is the one in the act of receiving the scolding. This is not the same with all helper verbs.

The receiver is not always a person:
\begin{quote}
水がの*ま*れた - Water got drunk
\end{quote}
Even if we add a doer of the drinking, the water is still the actor of the sentence.
\begin{quote}
水がいぬにの*ま*れた - Water got drunk by (a) dog
\end{quote}
いぬに is modifying のむ.

Why is the dog being marked by に? Let's look at a larger sentence:
\begin{quote}
さくらは だれかに かばんが ぬす*ま* れた - As for Sakura, someone-by bag stolen-got - As for Sakura, (her) bag got stolen by someone
\end{quote}

Who is the actor? It's not Sakura, she's marked by は. It's not the 'someone' as they're marked by に. The bag is the actor of the sentence, the bag \emph{did} 'got'.

What is に doing here? に marks the ultimate target of an action. So what is the target of getting stolen? To whome is the stolen item going? It is the 'someone' who stole it.

Note: Cure Dolly uses a \emph{push-pull} analogy here, which I think is unnecessary.

\subsection{The nuisance receptive}
\label{sec:org911ad0c}
\begin{quote}
さくらが だれかに かばんを ぬす*ま* れた
\end{quote}
Here the core of the sentence is now 'Sakura got'. What did she get? She got the unfortunate (nuisance) action of だれかに かばんを ぬすむ, someone stealing (her) bag. \emph{Sakura got her bag stolen by someone} \textbf{not} \emph{Sakura's bag got stolen by someone}.
\section{Lesson 14: Adverbs and も-particle}
\label{sec:org9b2b15c}
\begin{quote}
アリスは とび 上がって、 ウサギの 後を 追った。
\end{quote}
\begin{itemize}
\item とび上がって = とぶ (jump/fly) + 上がる (rise up) = Jump up
\item 後を追った = 後 (after) + 追う (follow) = Follow after/follow behind
\end{itemize}

Note how we are following ウサギの後 - \emph{the rabbit's after/behind}. We are following the back of the rabbit.
\begin{quote}
アリスは とび 上がって、 ウサギの 後を 追った。 - Alice jumped up and followed after the rabbit
\end{quote}

---

\begin{quote}
しゃべる ウサギを 見た ことが ない。
\end{quote}
Here しゃべる is being used as an adjective just as any verb can be. しゃべるウサギ - Talkative/talking rabbit.

見た is the passed tense of 見る to see. It is modifying こと, an abstract thing, meaning 'The fact of having seen'. 見たことがない means 'The fact of having seen doesn't exist'.

The talking rabbit is the object of the engine of the sentence, 'The fact of having seen'. So: 'The fact of having seen a talking rabbit doesn't exist' -> (Alice) had never seen a talking rabbit. This is another example of the un-egocentric nature of Japanese; Alice is not the actor of this sentence, it is the 'thing' that does not exist.

---

\begin{quote}
ウサギは 早く 走って、 急に ウサギの 穴に とび 込んだ。
\end{quote}
\begin{quote}
ウサギは 早く 走って、
\end{quote}
\begin{itemize}
\item 早い - fast/early
\item 走る - run
\end{itemize}
早い is an adjective. If we want to say the rabbit is fast we simply say ウサギが早い. But if we want to say that the rabbit's \textbf{movement} is fast we must use an adverb. In Japanese we can turn any adjective into an adverb by simply removing the い and replacing it with く. 早い -> 早く.

Note again how our verb 走る has been converted to the て form signifying an 'and'.

\begin{quote}
急に ウサギの 穴に とび 込んだ。
\end{quote}
\begin{itemize}
\item ウサギの 穴に - Here we aren't saying 'The rabbit's hole' but 'a hole of the variety rabbit'.
\item とび 込む = とぶ (jump) + こむ (Squeeze (?) into) = Jump into
\end{itemize}

急に is another way of forming an adverb. 急 is a noun meaning 'suddenly'. We can form an adjective from a noun by adding に.

\begin{quote}
ウサギは 早く 走って、 急に ウサギの 穴に とび 込んだ。- The rabbit ran quickly and jumped into a rabbit hole.
\end{quote}

\subsection{The も flag}
\label{sec:orgd749687}
\begin{quote}
アリス*も* ウサギの 穴に とび こんだ。 - Alice also jumped into the rabbit hole.
\end{quote}

も is another non-logical topic-marking particle. も marks the topic of the sentence in the same way that は does. The difference is that while は can mark the topic of the sentence and \textbf{can} also change the topic of the sentence, も decalares the topic of the sentence but can \textbf{only} change the topic of the sentence. We cannot use も unless we are changing the topic of the sentence. Up until this point the topic of our conversation has been the rabbit, now we are switching to talk about Alice.

When we change topic with も we're saying that the comment about the previous topic (the rabbit and that it jumped) is the same as our new topic (Alice). When we change the topic with は we are doing the opposite, we are drawing a distinction between the two.

---

\begin{quote}
穴の 中は たて穴 だった。 アリスは すぐ下に 落さた。- The inside of the hole was a vertical hole. Alice fell straight (directly) down the hole.
\end{quote}

\begin{quote}
でも、 おどろいたことに ゆっくり ゆっくり 落さた。 - But, the surprising thing was that she slowly slowly fell/ But, surprisingly she fell slowly.
\end{quote}
\begin{itemize}
\item おどろいたこと doesn't mean 'A surprised thing', it means 'the surprising thing' (surprisingly). The に attached is again to turn it into an adverb. So: 'She fell surprisingly'. Of course, it isn't surprising that she fell, but it is suprising that she fell ゆっくり ゆっくり (slowly slowly).
\item ゆっくり is slightly unusual in that it is fundamentall a noun, but we can use it as an adjective without adding に to it. We will see ゆっくり very often.
\end{itemize}


\section{Lesson 15: Transitivity}
\label{sec:org473fdf6}
\emph{Transitive} and \emph{intransitive} aren't as big of a misnomer as some of the things we've seen so far, but a better pair of terms would be \emph{Self-move} and \emph{Other-move}.

In japanese, a move-word 動詞 (どうし) is a word that denotes an action of a movement. So a self-move verb is a verb that moves itself. If I 'stand-up' that's a self-move action. But throwing a ball is an 'other-move' action, one is not throwing themselves, they are throwing a ball. It's as simple as that.

Japanese has a lot of pairs of words, these could be called forms, or just closely related words, that give the self-move and other-move variations of the verb. For example:
\begin{itemize}
\item 出る (でる) - leave, exit, come out - Self-move
\item 出す (だす) - take out, bring out - Other-move
\end{itemize}

Most of the time we can tell which is a self-move word and which is an other-move word by following a few simple rules.

The first thing to know is that there is a root word for self-move and a root word for other-mode:
\begin{itemize}
\item ある - Be - Self-move
\item する - Do - Other-move
\end{itemize}

Knowing this there are three laws of move-word pairs.
\begin{enumerate}
\item す and せる (え-stem) ending verbs are other-move
\item あ-stem + る (aru) ending verbs are self-move
\item え-stem + る (eru) flip self/other-move either way
\end{enumerate}

Honorary members of the す family:
\begin{itemize}
\item む -> める is always other-move
\item ぶ -> べる is always other-move
\item つ -> てる is always other-move
\end{itemize}

The only wildcards left are:
\begin{itemize}
\item く/ぐ -> ける/げる
\item う ー> える
\item Some る-ending verbs not covered by the first two laws
\end{itemize}

Is there anything we can do to simplify this: える version have the opposite of the standard word.
\end{document}
